%%%%%% LaTeX-Vorlagedatei für die Bachelorarbeit
%%%%%% der Fakultät für Bau- und Umweltingenieurwesen der TU Wien
%%%%%% Die Datei BachelorBUI.cls muss im gleichen Ordner liegen
%%
%% created by Christian Schranz and Sebastian Pech
%% CEE Computer Lab & Digital Building Process
%% date: 2022-10-01
%% tested for pdflatex
%%
\documentclass{BachelorBUI}
%%
%% in oberer Dokumentklasse bereits geladene Pakete
%%
%% fontenc[T1], lmodern, microtype, babel[englisch,ngerman], graphicx, 
%% geometry with all margins or areaset (choose what you like)
%% mathtools, amssymb, xfrac, siunitx, booktabs,
%% url, xcolor[table], textcomp, marvosym, pifonts, pdfpages, ragged2e, 
%% tabularx, longtable, threeparttable, csquotes, eurosym, enumitem, 
%% multirow, setspace, listings, scrlayer-scrpage with header/footline
%% pdfx (including hyperref)
%%
\usepackage[utf8]{inputenc}
\RequirePackage[babel,austrian=quotes,english=american]{csquotes}   %% context sensitives quotations

\raggedbottom 

\lstset{
	language={Matlab},
}

\sisetup{output-decimal-marker = {,},
range-phrase = --,
group-separator = {~},
per-mode = symbol, 
list-final-separator={ und }}

\graphicspath{{Bilder/}}

\newcommand{\zB}{\mbox{z.\,B.}\xspace}
\newcommand{\Name}[1]{\textsc{#1}}

\newcommand{\vKTxv}{\mathbf{v}_1^T\tilde{\mathbf{K}}_{T},_{\xi}\mathbf{v}_1}
\newcommand{\vKTxxv}{\mathbf{v}_1^T\tilde{\mathbf{K}}_{T},_{\xi\xi}\mathbf{v}_1}

%% biblatex and biber für das Literaturverzeichnis
%% verschiedene Zitiervarianten über Optionen einstellen:
%% style=numeric-comp ... [1]
%% style=authoryear ... Mang 1998 / use \textcite{} ... Mang (1998)
%%
\usepackage[style=numeric-comp,backend=biber,maxcitenames=2]{biblatex}
\ExecuteBibliographyOptions{%
  giveninits=true,maxbibnames=99}%
\DefineBibliographyStrings{ngerman}{%
andothers={et\;al\adddot},
urlseen = {Zugriff am}
}
\addbibresource{examples/Literatur.bib}

\usepackage{acro}
\acsetup{list/display=used}

\input{Acronyms}

% url-Trennregeln:
\setcounter{biburllcpenalty}{9000}% Kleinbuchstaben
\setcounter{biburlucpenalty}{9000}% Großbuchstaben
% (siehe https://texwelt.de/fragen/7008/zeilenumbruche-in-bibliografielinks )

%% Nun geben Sie einen Titel ein
\title{Instruktionen zur Abfassung der Bachelorarbeit}  
\authorname{Vorname Nachname}  	%% Hier fürgen Sie Ihren Namen ein
\email{email@email.at}  			%% Hier Ihre E-Mailadresse angeben
\MatrNr{12345678}   					%% Hier Matrikelnummer angeben
\thesislanguage{de-AT}                        %% Sprache der Arbeit (de-AT, de-DE, en-GB, en-US)
\keywords{bachelor thesis\sep template\sep LaTeX}
%% Package pdfx derzeit fehlerhaft implementiert ... daher pdfx auskommentiert und stattdessen hyperref geladen
%\usepackage[a-2u,pdf15]{pdfx}
% hier beginnt das Dokument
\begin{document}
\selectlanguage{ngerman}
%% ========== Key Words ==========
%%
%% Geben Sie beim Befehl \Keywords einige Key Words ein und 
%% trennen Sie diese mit dem Befehl \sep%%
\begin{filecontents}[overwrite]{\jobname.xmpdata}
\makeatletter
\Title{\@title}
\Author{\@authorname}
\Language{\@thesislanguage}
\Keywords{\@keywords}
\Publisher{TU Wien}
\makeatother
\end{filecontents}
%%%%%%%%%
%% Titel
%%%%%%%%%
%
\maketitle

\begin{abstract}

Die Kurzfassung soll den Inhalt der Arbeit kurz zusammenfassen. 
Sie sollte zumindest 70 und maximal 150~Wörter beinhalten. 
Der Schriftgrad sollte 10-Punkt sein. 
Der Einzug links und rechts soll \SI{1,0}{cm} betragen.
Der Text in der Kurzfassung wird innerhalb der Umgebung 
\texttt{\textbackslash{}begin\{abstract\}} und 
\texttt{\textbackslash{}end\{abstract\}} geschrieben.

\end{abstract}

% Bei Bedarf Inhaltsverzeichnis aktivieren:
\tableofcontents

\section{Einleitung}

Die Bachelorarbeit kann in Deutsch oder Englisch verfasst werden. 
Die Länge darf 12~Seiten nicht unterschreiten und 30~Seiten nicht 
überschreiten (exkl. Anhang).
Nach dem Titel der Arbeit werden der Autor und darauf eine Kurzfassung 
angeführt. 
Danach beginnt der Hauptteil der Arbeit. 
Die Bachelorarbeit hat keine Titelseite und nur bei Bedarf ein 
Inhaltsverzeichnis (zwischen Kurzfassung und Kapitel~1).

Der Titel der Arbeit wird in den Befehl \texttt{\textbackslash{}title\{\}} 
geschrieben,
der Name des Autors in den Befehl \texttt{\textbackslash{}studentname\{\}},
die E-Mailadresse in den Befehl \texttt{\textbackslash{}email\{\}} und
die Matrikelnummer in den Befehl \texttt{\textbackslash{}MatrNr\{\}}.
Der Befehl \texttt{\textbackslash{}maketitle} generiert den Titelblock mit 
allen nötigen Informationen und wird unmittelbar nach dem Befehl 
\texttt{\textbackslash{}begin\{document\}} gesetzt.

\section{Bachelorarbeit}

Nach der Einleitung kommt der weitere Text:  
Text Text Text Text Text Text Text Text Text Text Text Text Text Text Text 
Text Text Text Text Text Text Text.

\subsection{Textbereich}

Der Text sollte die ganze Breite ausfüllen, also im Blocksatz stehen. 
Die Silbentrennung soll aktiviert sein. 
Der erste Absatz ist nicht eingerückt, die folgenden dann schon.

\LaTeX{} generiert dieses Einrücken automatisch.

\subsection{Layout, Schriftart, Schriftgrad und Nummerierung}

Für diese Arbeit wird die Dokumentklasse \texttt{BachelorBI} verwendet, die
auf der KOMA-Skript-Klasse \texttt{scrartcl} aufbaut.
Der Haupttext ist in 11-Punkt-Schrift und einzeiligem Abstand geschrieben. 
Die anderen Schriftgrade passen sich automatisch an. 
Die Formatierung Kursiv kann für Hervorhebungen verwendet werden. 
Von der Formatierung Fett und Unterstrichen wird jedoch abgeraten. 

Es sollten in keinem Fall mehr als drei Überschriftsebenen verwendet werden.

\begin{table}
\caption{Die Tabellenbeschriftung ist normalerweise über der Tabelle. 
Tabellen sollen durchgehend nummeriert sein. 
Der letzte Satz der Tabellenbeschriftung endet ohne Punkt \label{tab:1}}
\centering{
\begin{tabular}{lll}
\toprule
Überschriftenebene & Beispiel         & \LaTeX{}-Befehle \\
\midrule
Titel (zentriert)  & Instruktionen    & \textbackslash{}title\{\} \\
Überschrift 1      & 1 Einleitung     & \textbackslash{}section\{\} \\
Überschrift 2      & 2.1 Textbereich  & \textbackslash{}subsection\{\} \\
Überschrift 3      & 2.1.1 Unterlagen & \textbackslash{}subsubsection\{\} \\
\bottomrule
\end{tabular}
}
\end{table}

Die Tabelle sowie deren Beschriftung werden innerhalb der Umgebung 
\texttt{\textbackslash{}begin\{table\}} und 
\texttt{\textbackslash{}end\{table\}} geschrieben.
Die Beschriftung der Tabelle wird in den Befehl 
\texttt{\textbackslash{}caption\{\}} geschrieben, dieser ist vor 
\texttt{\textbackslash{}begin\{tabular\}} zu positionieren.
Das Erstellen der Tabelle erfolgt innerhalb der Umgebung 
\texttt{\textbackslash{}begin\{tabular\}} und 
\texttt{\textbackslash{}end\{tabular\}}.

\subsection{Abbildungen und Fotos}

Abbildungen sollten digital erstellt sein (keine Handzeichnungen). 
Diese müssen dann in der Bachelorarbeit integriert sein.
Diagramme sollen gut lesbar erstellt werden. 
Der Schriftgrad innerhalb der Abbildungen soll 10-Punkt sein.

Abbildungen sollen durchgehend nummeriert sein. 
Die Abbildungsbeschriftung ist immer unterhalb der Abbildung. 
Der letzte Satz der Abbildungsbeschriftung endet ohne Punkt. 
Die Abbildungsbeschriftung soll 10-Punkt-Schrift besitzen. 
Ein Beispiel sehen Sie in Abb.~\ref{fig:kurven}.

\begin{figure}[h]
\centering{
\includegraphics[width=9cm]{tuw-bui-logo-2022}
}
\caption{Dies ist ein Beispiel für eine Abbildungsbeschriftung}
\label{fig:kurven}
\end{figure}

Das Bild sowie dessen Beschriftung werden innerhalb der Umgebung 
\texttt{\textbackslash{}begin\{figure\}} und 
\texttt{\textbackslash{}end\{figure\}} geschrieben.
Das Einfügen des Bildes erfolgt durch den Befehl 
\texttt{\textbackslash{}includegraphics\{\}}.
Die Beschriftung der Abbildung wird in den Befehl 
\texttt{\textbackslash{}caption\{\}} geschrieben, dieser ist nun nach 
\texttt{\textbackslash{}includegraphics\{\}} zu positionieren.

\subsection{Formeln}

Gleichungen und Formeln sollen generell zentriert in einer eigenen Zeile 
platziert werden. 
Die Gleichungen sollen durchnummeriert werden, wobei die Gleichungsnummer 
in Klammer zu stehen hat, z.B. Formeln werden in die Umgebung 
\texttt{\textbackslash{}begin\{equation\}} und 
\texttt{\textbackslash{}end\{equation\}} oder ähnlichen Umgebungen 
geschrieben.

\begin{equation}
K_t
=
\left(
1-
\frac{R^2\cdot\tau}{c_a+\nu\cdot\tan\delta}
\right)^4 \cdot k_1
\label{Glg:Kt}
\end{equation}

\subsection{Program Code}

Program Code im Textbereich wird normalerweise mit einer 
Schreibmaschinenschrift formatiert.
Program Codes werden in die Umgebung 
\texttt{\textbackslash{}begin\{lstlisting\}} und 
\texttt{\textbackslash{}end\{lstlisting\}} geschrieben.

In Program Code~\ref{code} sehen Sie ein 50-Zeilen-FEM-Programm in Matlab von \textcite{Alberty:1999}.
In diesem relativ kurzem Programm versteckt sich ein vollwertiges Finite-Element-Programm zur Berechnung des 2-dimensionalen Laplace Problems.

\begin{lstlisting}[firstnumber=1,caption={50 Lines of Matlab},label={code}]
%FEM2D two-dimensional finite element method for Laplacian.
% Initialisation
load coordinates.dat; coordinates(:,1)=[];
eval('load elements3.dat; elements3(:,1)=[];','elements3=[];');
eval('load elements4.dat; elements4(:,1)=[];','elements4=[];');
eval('load neumann.dat; neumann(:,1) = [];','neumann=[];');
load dirichlet.dat; dirichlet(:,1) = [];
FreeNodes=@setdiff@(1:size(coordinates,1),unique(dirichlet));
A = sparse(size(coordinates,1),size(coordinates,1));
b = sparse(size(coordinates,1),1);

% Assembly
for j = 1:size(elements3,1)
  A(elements3(j,:),elements3(j,:)) = A(elements3(j,:),elements3(j,:)) ...
      + stima3(coordinates(elements3(j,:),:));
end
for j = 1:size(elements4,1)
  A(elements4(j,:),elements4(j,:)) = A(elements4(j,:),elements4(j,:)) ...
      + stima4(coordinates(elements4(j,:),:));
end

% Volume Forces
for j = 1:size(elements3,1)
  b(elements3(j,:)) = b(elements3(j,:)) + ...
      det([1,1,1; coordinates(elements3(j,:),:)']) * ...
      f(sum(coordinates(elements3(j,:),:))/3)/6;
end
for j = 1:size(elements4,1)
  b(elements4(j,:)) = b(elements4(j,:)) + ...
      det([1,1,1; coordinates(elements4(j,1:3),:)']) * ...
      f(sum(coordinates(elements4(j,:),:))/4)/4;
end

% Neumann conditions
for j = 1 : size(neumann,1)
  b(neumann(j,:))=b(neumann(j,:)) + norm(coordinates(neumann(j,1),:)- ...
      coordinates(neumann(j,2),:)) * 
      g(sum(coordinates(neumann(j,:),:))/2)/2;
end

% Dirichlet conditions 
u = sparse(size(coordinates,1),1);
u(unique(dirichlet)) = u_d(coordinates(unique(dirichlet),:));
b = b - A * u;

% Computation of the solution
u(FreeNodes) = A(FreeNodes,FreeNodes) \ b(FreeNodes);

% graphic representation
show(elements3,elements4,coordinates,full(u));
\end{lstlisting}

\subsection{Fußnoten}

Die Fußnotenziffer ist entweder direkt nach dem zu beschreibenden Wort oder 
nach einem Satzzeichen angeordnet.
Fußnoten werden in den Befehl \texttt{\textbackslash{}footnote\{\}}
geschrieben.

\subsection{Referenzierungen und Literaturverzeichnis}

Die Liste der Referenzierungen ist mit „Literatur“ betitelt und wird ohne 
eigenen Seitenumbruch am Ende der Arbeit positioniert (aber noch vor einem 
möglichen Anhang). 
Die zugehörige Überschrift hat dann keine Überschriftennummer. 
Das Literaturverzeichnis wird mit \texttt{biblatex} erstellt und mit dem Befehl \texttt{\textbackslash{}printbibliography} an der gewünschten Stelle eingefügt.

Die Literaturangaben sollen nummeriert angeführt werden. 
Die Nummerierung selbst steht innerhalb eckiger Klammern, 
z.B. \cite{Kohm:20}, \cite{Schranz:22}, \cite{Voss:22}, \dots oder \cite{Kohm:20,Schranz:22,Voss:22}

Die Quellenangabe kann auch hinter dem Zitat oder inhaltlicher Behauptung 
wie folgt angeführt werden: 
(Name des Erstautor et al. (Jahreszahl)), z.B. (Alberty et al. (1999)).
Dann muss ein geeigneter Literaturstil angegeben werden.

\subsection{Kopfzeile}

Die Bachelorarbeit ist für einen doppelseitigen Druck formatiert. 
Daher befindet sich in der Kopfzeile außen die Seitennummer 
(bei geraden Seiten links und bei ungeraden Seiten rechts) 
sowie innen der Titel der Arbeit 
(bei geraden Seiten rechts und bei ungeraden Seiten links).

Dies erfolgt in der Dokumentklasse \texttt{BachelorBI} automatisch.

% Hier wird das Literaturverzeichnis eingefügt.
\printbibliography

\minisec{Zitierbeispiele entsprechend DIN ISO 690}

Die DIN~ISO~690~\cite{DIN-ISO-690:2013} gibt Hinweise zur vollständigen Quellenangabe.
Das folgende Beispiele ist dieser Norm entlehnt und angepasst:
%
\begin{quotation}
Einige Standardwerke \cite{Kohm:20,Voss:22} geben einen guten Überblick über \LaTeX{}.
\textcite{Kohm:20} hat die Klassen der KOMA-Script-Reihe entwickelt, die die \enquote{typografischen Gepflogenheiten eines europäischen Layouts berücksichtigen}~\cite[S.~64]{Voss:22}.
Daneben empfehlen wir -- nona -- unser Buch zu \LaTeX{}, Excel und Word: \textcite{Schranz:22}.
\end{quotation}

Nun der selbe Text in Harvard Citation Style (jedoch ohne \LaTeX-Befehle -- diese wären die selben wie zuvor, es müsste nur der Zitierstil geändert werden):
%
\begin{quotation}
Einige Standardwerke (Kohm 2020, Voss 2022) geben einen guten Überblick über \LaTeX{}.
Kohm (2020) hat die Klassen der KOMA-Script-Reihe entwickelt, die die \enquote{typografischen Gepflogenheiten eines europäischen Layouts berücksichtigen}~(Voss 2022, S.~64).
Daneben empfehlen wir -- nona -- unser Buch zu \LaTeX{}, Excel und Word: Schranz et al. (2022).
\end{quotation}


\appendix

\addsec{Anhang}

Ein möglicher Anhang sollte direkt nach dem Literaturverzeichnis ohne 
Seitenumbruch angeführt werden.
Jede Anhangüberschrift wird durch den Befehl 
\texttt{\textbackslash{}addsec\{Anhang\}} 
erstellt (anstatt \texttt{\textbackslash{}section\{\}}).

\addsec{Anhang A -- Beispiele aus dem Video zur Bachelorarbeit}

In Anhang~A finden Sie nun die zuvor nicht behandelten Beispiele aus dem youtube-Video zur Bachelorarbeit:

\url{https://www.youtube.com/playlist?list=PLwlC-XZXtzhg4fQiZQAsXIMSRW-iZtnTQ}

\input{beispiele1.tex}



\addsec{Anhang B -- Beispiele aus dem Video zur DA und Diss}

In Anhang~B finden Sie einige neue Beispiele aus dem youtube-Video zu Diplomarbeiten und Dissertationen:

\url{https://www.youtube.com/playlist?list=PLwlC-XZXtzhg4fQiZQAsXIMSRW-iZtnTQ}

\minisec{Zwei Abbildungen nebeneinander}
%
\texttt{Abbildungen} fallen unter \emph{floating objects} und werden in der \texttt{figure}-Umgebung in den Text eingebunden.
Für mehrere Bilder in einer Abbildung mit jeweils eigener Beschriftung können wir die \texttt{subfigure}-Umgebung verwenden (siehe Abb.~\ref{fig:bsp-subfigure})
%
Abb.~\ref{fig:xkcd-citogenesis} zeigt ein Problem des Zitierens aus Wikipedia. 
Da manche Menschen alles glauben, was in Wikipedia geschrieben steht, stellt dieser Comik aus \url{https://xkcd.com} die Glaubwürdigkeit zumindest ein bisschen in Frage.
Eine Erklärung dazu befindet sich auf \url{https://www.explainxkcd.com/wiki/index.php/978:_Citogenesis}.
Abb.~\ref{fig:logo-fakultät} zeigt das Logo unserer Fakultät.
%
\begin{figure}[ht]
  \begin{subfigure}[t]{0.50\textwidth}
   \centering
   \includegraphics[width=7cm]{citogenesis}
   \caption[Zitierproblematik]{Ein Glaubwürdigkeitsproblem mancher Artikel in Wikipedia (Quelle: \url{https://xkcd.com/978/})    \label{fig:xkcd-citogenesis}}
  \end{subfigure}
\hfill
  \begin{subfigure}[t]{0.45\textwidth}
   \centering
   \includegraphics[width=0.85\textwidth]{tuw-bui-logo-2022}
   \caption{Logo der Fakultät für Bau- und Umweltingenieurwesen der TU Wien \label{fig:logo-fakultät}}
  \end{subfigure}
\caption{Beispiel einer \texttt{subfigure}-Umgebung \label{fig:bsp-subfigure}}
\end{figure}
%


\minisec{Matrix-Schreibweise}
%
Die Steifigkeitsmatrix sowie die Nachgiebigkeitsmatrix des Materials sind mit Bezug auf die Materialhauptrichtungen $L$-$R$-$T$ gegeben (siehe~\eqref{Glg:sigma} bis \eqref{Glg:D-matrix}).

Gesucht ist der Verzerrungstensor $\varepsilon$ im globalen Koordinatensystem $X,Y,Z$ sowie die Koordinaten der Eckpunkte in der verformten Lage (angegeben in [\si{mm}] auf 3~Dezimalstellen) unter der Annahme einer linearisierten Elastizitätstheorie.

\begin{equation}
\boldsymbol{\sigma}
=
\left(
   \begin{array}{c}
      \sigma_{xx} \\
      \sigma_{yy} \\
      \sigma_{zz} \\
      \sigma_{xy} \\
      \sigma_{yz} \\
      \sigma_{zx} \\
   \end{array}
\right)
=
\left(
   \begin{array}{c}
      1x{,}y \\
      1{,}xy \\
      2{,}xy \\
      3{,}xy \\
      0      \\
      0      \\
   \end{array}
\right)
\si{N/mm^2}
\label{Glg:sigma}
\end{equation}

\begin{equation}
\textbf{C}_{(LRT)}
=
\left(
\begin{array}{rrrccc}
\num{15500} & 489 & 274 & 0   & 0  & 0  \\
      & 844 & 162 & 0   & 0  & 0  \\
      &     & 632 & 0   & 0  & 0  \\
      &     &     & 700 & 0  & 0  \\
      &     &     &     & 60 & 0  \\
   \multicolumn{2}{l}{\text{symm.}}  
            &     &     &    & 650 \\
\end{array}
\right)
%
\cdot
%
\left(
   \begin{array}{c}
      \epsilon_{xx} \\
      \epsilon_{yy} \\
      \epsilon_{zz} \\
      \epsilon_{xy} \\
      \epsilon_{yz} \\
      \epsilon_{zx} \\
   \end{array}
\right)
=
\left(
   \begin{array}{r}
      \num{12,70} \\
      \num{1,27}  \\
      \num{2,27}  \\
      \num{3,27}  \\
      \multicolumn{1}{c}{0}     \\
      \multicolumn{1}{c}{0}     \\
   \end{array}
\right)
\label{Glg:C-matrix}
\end{equation}

\begin{equation}
\textbf{D}_{(LRT)}
=
\left(
   \begin{array}{rrrccc}
\num{6,595} &  \num{-3,441} &  \num{-1,977} &  0  &  0  & 0 \\
            & \num{126,411} & \num{-30,911} &  0  &  0  & 0 \\
            &               & \num{167,008} &  0  &  0  & 0 \\
            &               &               & \num{142,857} &  0  &  0  \\
            &               &               &    & \num{1666,667}  &  0  \\
   \multicolumn{2}{l}{\text{symm.}} &       &    &      & \num{153,846}  \\
\end{array}
\right)
\cdot \SI{e-5}{mm^2/N}
\label{Glg:D-matrix}
\end{equation}



\minisec{Beispiel für xfrac-Paket}
%
Dieses Paket verwendet den Befehl \texttt{\textbackslash{}sfrac\{\}\{\}} im Text
\sfrac{3}{4} oder in einer Formel $\sfrac{3}{4}$:
%
\begin{equation}
\sfrac{3}{5} \cdot x = y
\end{equation}

\minisec{Beispiele für den Einsatz des siunitx-packages}
Eine Verwendungsmöglichkeit ist die richtige Anzeige von Zahlen:
%
\begin{itemize} 
  \item als Einzelzahl: \num{12345678,9202}
  \item als Bereich von Zahlen: \numrange{12,3}{14,7} 
  \item als Liste von Zahlen: 
                      \numlist[list-final-separator={ und }]{12,3;13,5;14,7}
\end{itemize}
%
oder die richtige Darstellung von Einheiten (unabhängig ob im Paragraph- oder Math-Modus):
%
\begin{itemize} 
  \item Paragraphmodus: \si{\kilo\newton\per\meter}, \si{kN/m^2}
  \item Mathematikmodus:  $\si{\kilo\newton\per\meter}$, $\si{kN/m^2}$
\end{itemize}
%
oder die richtige Darstellung von Zahlen mit Einheiten:
%
\begin{itemize} 
  \item als Einzelzahl: \SI{12345678,92}{kNm}
  \item als Winkel: \ang{10}, \ang{12.3} oder \ang{12;3;5}
  \item als Bereich von Zahlen: \SIrange{12,3}{14,7}{\%} oder 
          \SIrange[range-units = single]{12,3}{14,7}{\%}
  \item als Liste von Zahlen: 
        \SIlist[list-final-separator={ und }]{12,3;13,5;14,7}{\kilogram/m^2}
\end{itemize}
%

\minisec{Beispiel für den Einsatz des tabularx-packages}
Dieses Paket bietet die Möglichkeit der automatischen Anpassung der Spaltenbreite auf eine Gesamtbreite der Tabelle (siehe Tab.~\ref{tab:test}).
%
\begin{table}[h]
\caption{Ergebnisse der schriftlichen Prüfung \label{tab:test}}
   \begin{tabularx}{\textwidth}{@{}lccX@{}}
   \toprule
   Name & Entwurf     & Pläne       & Anmerkung    \\
   \midrule
   Mayer    & \SI{60}{\%} & \SI{60}{\%} & 
      Funktionale Umsetzung mit hinreichender Routine. Konstruktive Darstellung speziell im Dachbereich nicht nachvollziehbar. \\
   Müller   & \SI{20}{\%} & \SI{30}{\%} & 
      In allen Teilbereichen sehr detailierte Konzeption, allerdings fehlt die planliche Umsetzung, sodass aufgrund des fehlenden Informationsgehalts keine positive Beurteilung möglich ist. \\
   Schmidt  & \SI{90}{\%} & \SI{90}{\%} & 
      In allen Prüfungsabschnitte routinierte Darstellung und planliche Umsetzung. Die Nachvollziehbarkeit ist in allen Teilabschnitten gegeben. \\
   \bottomrule
   \end{tabularx}
\end{table}



\minisec{Beipiele für Abkürzungen samt zugehörigem Verzeichnis}

Das Programm \ac{LZKB} wurde in Zusammenarbeit mit der \ac{obv} erstellt, berücksichtigt die deutsche \ac{abbv} -- von mehreren \aclp{abbv} -- und berechnet \ac{lzk}.

Nochmals:
Das Programm \ac{LZKB} wurde in Zusammenarbeit mit der \ac{obv} erstellt, berücksichtigt die deutsche \ac{abbv} -- von mehreren \acp{abbv} -- und berechnet \ac{lzk}.

\printacronyms[name=Abkürzungen]






\minisec{Beispiel für den Einsatz der longtable- und multirow-packages}

\begin{longtable}{@{}l*{3}{S[table-format=2.2]}@{}}
\caption{Messwerte der bauphysikalischen Untersuchung \label{tab:bauphysik}}
\\ \toprule
Datum & {Mittel [\si{\degreeCelsius}]} & {Min [\si{\degreeCelsius}]} & {Max [\si{\degreeCelsius}]}
\\ \midrule
\endfirsthead
\caption{Messwerte der bauphysikalischen Untersuchung (Fortsetzung)}
\\ \toprule
Datum & {Mittel [\si{\degreeCelsius}]} & {Min [\si{\degreeCelsius}]} & {Max [\si{\degreeCelsius}]}
\\ \midrule
\endhead
%
  \midrule
  \multicolumn{4}{r}{{Continued on next page}} 
  \\ \bottomrule
\endfoot
%
  \bottomrule
\endlastfoot
06-Jul-2016 & 24,6 & 20,9 & 25,0 \\
07-Jul-2016 & 24,82 & 24,50 & 25,30 \\
08-Jul-2016 & 24,58 & 24,30 & 25,10 \\
09-Jul-2016 & 24,58 & 24,40 & 24,80 \\
10-Jul-2016 & 24,53 & 24,40 & 24,90 \\
11-Jul-2016 & 24,55 & 24,20 & 25,00 \\
12-Jul-2016 & 24,55 & 24,40 & 24,70 \\
13-Jul-2016 & 24,55 & 24,40 & 24,70 \\
14-Jul-2016 & 25,02 & 24,50 & 25,40 \\
15-Jul-2016 & 25,22 & 24,80 & 25,50 \\
16-Jul-2016 & 25,45 & 25,30 & 25,70 \\
17-Jul-2016 & 25,35 & 24,80 & 25,70 \\
18-Jul-2016 & 25,29 & 24,80 & 25,70 \\
19-Jul-2016 & 24,83 & 24,60 & 25,10 \\
20-Jul-2016 & 24,73 & 24,60 & 25,00 \\
21-Jul-2016 & 24,68 & 24,50 & 24,90 \\
22-Jul-2016 & 24,70 & 24,60 & 25,00 \\
23-Jul-2016 & 24,72 & 24,60 & 25,00 \\
24-Jul-2016 & 24,81 & 24,60 & 25,00 \\
25-Jul-2016 & 24,74 & 24,50 & 25,00 \\
26-Jul-2016 & 24,70 & 24,60 & 24,80 \\
27-Jul-2016 & 24,72 & 24,50 & 25,00 \\
28-Jul-2016 & 24,66 & 24,50 & 24,90 \\
29-Jul-2016 & 24,66 & 24,50 & 24,80 \\
30-Jul-2016 & 24,69 & 24,60 & 24,80 \\
31-Jul-2016 & 24,77 & 24,70 & 25,00 \\
01-Aug-2016 & 24,72 & 24,40 & 25,00 \\
02-Aug-2016 & 24,64 & 24,50 & 24,90 \\
03-Aug-2016 & 24,73 & 24,60 & 25,00 \\
04-Aug-2016 & 24,74 & 24,60 & 24,80 \\
05-Aug-2016 & 24,76 & 24,60 & 25,20 \\
06-Aug-2016 & 25,30 & 24,80 & 25,70 \\
07-Aug-2016 & 25,10 & 24,70 & 25,60 \\
08-Aug-2016 & 25,06 & 24,80 & 25,50 \\
09-Aug-2016 & 24,89 & 24,70 & 25,20 \\
10-Aug-2016 & 25,52 & 25,00 & 25,80 \\
11-Aug-2016 & 25,60 & 25,30 & 25,80 \\
12-Aug-2016 & 25,81 & 25,60 & 26,00 \\
13-Aug-2016 & 25,95 & 25,50 & 26,30 \\
14-Aug-2016 & 25,79 & 25,40 & 26,30 \\
15-Aug-2016 & 25,47 & 25,20 & 25,90 \\
16-Aug-2016 & 25,36 & 25,10 & 25,90 \\
17-Aug-2016 & 25,33 & 25,10 & 25,80 \\
18-Aug-2016 & 25,42 & 25,00 & 25,90 \\
19-Aug-2016 & 25,36 & 25,00 & 25,70 \\
20-Aug-2016 & 25,37 & 25,00 & 25,80 \\
21-Aug-2016 & 25,38 & 25,10 & 25,70 \\
22-Aug-2016 & 25,88 & 25,60 & 26,20 \\
23-Aug-2016 & 25,85 & 25,10 & 26,70 \\
24-Aug-2016 & 25,10 & 24,80 & 26,30 \\
25-Aug-2016 & 25,22 & 24,80 & 25,70 \\
26-Aug-2016 & 24,91 & 24,70 & 25,30 \\
27-Aug-2016 & 24,75 & 24,60 & 25,00 \\
28-Aug-2016 & 24,74 & 24,60 & 25,00 \\
29-Aug-2016 & 24,76 & 24,50 & 25,00 \\
30-Aug-2016 & 24,77 & 24,60 & 25,10 \\
31-Aug-2016 & 24,87 & 24,60 & 25,30 \\
01-Sep-2016 & 24,97 & 24,60 & 25,60 \\
02-Sep-2016 & 24,78 & 24,60 & 25,10 \\
03-Sep-2016 & 24,95 & 24,70 & 25,40 \\
04-Sep-2016 & 24,91 & 24,70 & 25,20 \\
05-Sep-2016 & 25,21 & 24,80 & 25,70 \\
06-Sep-2016 & 25,81 & 25,40 & 26,40 \\
07-Sep-2016 & 26,01 & 25,50 & 26,50 \\
08-Sep-2016 & 25,66 & 25,30 & 26,30 \\
09-Sep-2016 & 25,29 & 25,10 & 25,40 \\
10-Sep-2016 & 25,20 & 25,10 & 25,30 \\
11-Sep-2016 & 25,19 & 25,00 & 25,40 \\
12-Sep-2016 & 25,08 & 24,80 & 25,40 \\
13-Sep-2016 & 24,92 & 24,80 & 25,00 \\
14-Sep-2016 & 24,81 & 24,60 & 24,90 \\
15-Sep-2016 & 24,76 & 24,50 & 25,00 \\
16-Sep-2016 & 24,94 & 24,80 & 25,20 \\
17-Sep-2016 & 25,04 & 24,80 & 25,50 \\
18-Sep-2016 & 25,44 & 25,10 & 25,80 \\
19-Sep-2016 & 25,70 & 25,40 & 26,00 \\
20-Sep-2016 & 25,85 & 25,70 & 26,10 \\
21-Sep-2016 & 25,92 & 25,70 & 26,20 \\
22-Sep-2016 & 25,94 & 25,70 & 26,20 \\
23-Sep-2016 & 25,87 & 25,50 & 26,30 \\
24-Sep-2016 & 26,01 & 25,60 & 26,40 \\
\multirow{3}{*}{25--27-Sep-2016} 
            & 26,00 & 25,60 & 26,30 \\
            & 26,19 & 25,70 & 26,50 \\
            & 26,19 & 25,50 & 26,50 \\
\end{longtable}

\minisec{Beispiel für den Einsatz des threeparttable-packages}

\begin{table}[htpb]
  \centering
  \caption{Beispiel für einen threeparttable}
  \label{tab:near_optimal}
  \begin{threeparttable}
    \begin{tabular}{@{}l|lllll@{}}\toprule
      Location\tnote{1}             & Beam 1    & Beam 2    & Beam 3    & Beam 4    & Beam 5    \\ \midrule
      1                             & 16$^\ast$ & 21        & 28        & 32        & 36$^\ast$ \\
      2                             & 14$^\ast$ & 33        & 47        & 37$^\ast$ & 35$^\ast$ \\ \midrule
      Deflection [$10^{-5}\si{mm}$] & $4.4753$  & $4.4575$  & $4.5067$  & $4.5076$  & $4.4642$  \\
    \bottomrule\end{tabular}
    \begin{tablenotes}
    \item[] Lamella IDs marked with an asterisk are flipped
    \item[1] The location is defined from top to bottom of the beam.
    \end{tablenotes}
  \end{threeparttable}
\end{table}





\addsec{Anhang C -- Tipps zum guten Schreiben}

Die folgenden Tipps und Zitate sind direkt dem gut verständlichen Buch von Wolf Schneider~\cite{Schneider:11} entnommen. 
Sie sind für alle sehr wertvoll, die eine lesbare Arbeit schreiben möchten.
Schneider formuliert dies wie folgt:
%
\begin{quotation}
\emph{
Auf der Basis der korrekten Grammatik muss ich eine Kunst erlernen, die in der Schule ignoriert worden ist: \emph{wie man für Leser schreibt}. \dots 
\newline
Ganz ohne Plage geht das nicht.}
\end{quotation}

Die folgenden Ratschläge gelten für alle Texte, die nicht nur von einer Person gelesen werden sollen. 
Sie stammen direkt aus~\cite{Schneider:11}:
%
\emph{
\begin{itemize}
   \item Was für alle Texte gilt:
      \begin{itemize}
         \item Einfach draufloszuschreiben ist ein Luxus, den sich keiner
               leisten kann, der von Unbekannten gelesen werden möchte.
               Um seine Leser muss geworben werden, und ganz ohne Plage geht das nicht.
               Am Anfang stehen ein paar schlichte Einsichten -- und der Wille, sich an sie zu halten.
         \item Schon mit den ersten 350 Zeichen kann alles verdorben werden:
               wenn sie nämlich hingehudelt, wenn sie langweilig sind.
         \item Flickwörter müssen gemieden werden, hohle Redensarten ebenso.
               Kürze ist aber \emph{nicht} der oberste Wert: Bilder, Beispiele, Vergleiche sind oft nötig und stets willkommen.
         \item Wenn die Stillehrer sich in einem Punkt einig sind, dann in
               diesem: Der sicherste Weg, die Aufmerksamkeit der Lesenden zu wecken und wachzuhalten, ist der, besonders, bestimmt und konkret zu sein.
      \end{itemize}
   \item Das pralle Wort
         \begin{itemize}
            \item Konkrete, knackige, möglichst kurze Wörter sollten gewählt
                  werden; und in jedem Grenzfall die schlichtesten -- nach Schopenhauers schönem Satz: \emph{Man gebrauche gewöhnliche Worte und sage ungewöhnliche Dinge}.
            \item \emph{Verben} sind die stärksten Wörter, \emph{Adjektive} die
                  schwächsten. 
                  \emph{Synonyme} soll man für die Nebensachen wählen -- für die Hauptsachen nie.
            \item \emph{Anglizismen} gibt es gute und schlechte.
            \item Der größte Feind einer klaren, leserfreundlichen Sprache ist
                  der \emph{akademisch-bürokratische Jargon}.
         \end{itemize}
   \item Die elf Gebote des Satzbaus
         \begin{itemize}
            \item Verboten ist, 
               \begin{enumerate}
                  \item mehr als 6 Wörter zwischen die Teile eines zweiteiligen
                        Verbums oder zwischen Subjekt und Prädikat zu schieben.
                  \item einen Satz länger zu machen, als bei lautem Lesen der
                        Atem reichen würde.
               \end{enumerate}  
               \item Extrem unerwünscht sind 
               \begin{enumerate}
                  \item[3.] eingeschobene Nebensätze.
                  \item[4.] vorangestellte Attribute.
               \end{enumerate}
               \item Unerwünscht sind 
                \begin{enumerate}
                   \item[5.] hartnäckig wiederkehrende Satzanfänge.
                   \item[6.] atemlos gesetzte Punkte.
                \end{enumerate}
               \item \textbf{Erwünscht} sind 
                \begin{enumerate}
                   \item[7.] Hauptsätze -- immer die erste Wahl.
                   \item[8.] oft: angehängte Nebensätze -- ohne Handlung und mit Augenmaß.
                   \item[9.] manchmal: vorangestellte Nebensätze, wenn sie kurz sind.
                   \item[10.] Sätze wie Pfeile, vorwärtsstrebend.
                   \item[11.] mind. 5 der 7~Satzzeichen in \emph{jedem} Text. \newline
                              Diese 7 sind: Punkt, Komma, Doppelpunkt, Fragezeichen, Ausrufezeichen, Semikolon und Gedankenstrich.
                \end{enumerate}
         \end{itemize}
\end{itemize}
}
%
Bei Diplom- und Doktorarbeiten kann ein bisserl von obigen abgewichen werden.
So fängt eine dieser Arbeiten normalerweise mit einer Übersicht über Ihr Vorhaben sowie jenen Fragestellungen an, die Sie im Lauf der nächsten 80 bis 200~Seiten beantworten möchten.


\end{document}