%%%%%% LaTeX-Vorlagedatei für die Bachelorarbeit
%%%%%% der Fakultät für Bauingenieurwesen der TU Wien
%%%%%% Die Datei BachelorBUI.cls muss im gleichen Ordner liegen
\documentclass{BachelorBUI}
%
% in oberer Dokumentklasse bereits geladene Pakete
%
% fontenc[T1], lmodern, microtype, babel[englisch,ngerman], graphicx, 
% geometry with all margins or areaset (choose what you like)
% mathtools, amssymb, xfrac, siunitx, booktabs,
% url, xcolor[table], textcomp, marvosym, pifonts, pdfpages, ragged2e, 
% tabularx, longtable, threeparttable, csquotes, eurosym, enumitem, 
% multirow, setspace, listings, scrlayer-scrpage with header/footline

\usepackage[utf8]{inputenc}
\RequirePackage[babel,austrian=quotes,english=american]{csquotes}   %% context sensitives quotations

\raggedbottom 

\lstset{
	language={Matlab},
}

\sisetup{output-decimal-marker = {,},
range-phrase = --,
group-separator = {~},
per-mode = symbol, 
list-final-separator={ und }}

\graphicspath{{Bilder/}}

\newcommand{\zB}{\mbox{z.\,B.}\xspace}
\newcommand{\Name}[1]{\textsc{#1}}

\newcommand{\vKTxv}{\mathbf{v}_1^T\tilde{\mathbf{K}}_{T},_{\xi}\mathbf{v}_1}
\newcommand{\vKTxxv}{\mathbf{v}_1^T\tilde{\mathbf{K}}_{T},_{\xi\xi}\mathbf{v}_1}

%% biblatex and biber für das Literaturverzeichnis
%% verschiedene Zitiervarianten über Optionen einstellen:
%% style=numeric-comp ... [1]
%% style=authoryear ... Mang 1998 / use \textcite{} ... Mang (1998)
%%
\usepackage[style=numeric-comp,backend=biber,maxcitenames=2]{biblatex}
\ExecuteBibliographyOptions{%
  giveninits=true,maxbibnames=99}%
\DefineBibliographyStrings{ngerman}{%
andothers={et\;al\adddot},
urlseen = {Zugriff am}
}
\addbibresource{Literatur.bib}

\usepackage{acro}
\acsetup{only-used=true,extra-style=plain}

\input{Acronyms}


%% Nun geben Sie einen Titel ein
\title{Schreibtipps}  
\studentname{Vorname Nachname}  	%% Hier fürgen Sie Ihren Namen ein
\email{email@email.at}  			%% Hier Ihre E-Mailadresse angeben
\MatrNr{12345678}   					%% Hier Matrikelnummer angeben
% hier beginnt das Dokument
\begin{document}
\selectlanguage{ngerman}
%%%%%%%%%
%% Titel
%%%%%%%%%
%
\maketitle





\minisec{Zitierbeispiele entsprechend DIN ISO 690}

Die DIN~ISO~690~\cite{DIN-ISO-690:2013} gibt Hinweise zur vollständigen Quellenangabe.
Das folgende Beispiele ist dieser Norm entlehnt und angepasst:
%
\begin{quotation}
Sie sollten \textcite{Kornmeier:2018} lesen, bevor Sie zu schreiben beginnen.
Einige Standardwerke \cite{Kohm:2020,Voss:2017,Voss:2018} geben einen guten Überblick über \LaTeX{}.
\textcite{Kohm:2020} hat die Klassen der KOMA-Script-Reihe entwickelt, die die \enquote{typografischen Gepflogenheiten eines europäischen Layouts berücksichtigen}~\cite[S.~61]{Voss:2017}.
\end{quotation}

Nun der selbe Text in Harvard Citation Style (jedoch ohne \LaTeX-Befehle -- diese wären die selben wie zuvor, es müsste nur der Zitierstil geändert werden):
%
\begin{quotation}
Sie sollten Kornmeier (2018) lesen, bevor Sie zu schreiben beginnen.
Einige Standardwerke (Kohm 2020, Voss 2017, Voss 2018) geben einen guten Überblick über \LaTeX{}.
Kohm (2020) hat die Klassen der KOMA-Script-Reihe entwickelt, die die \enquote{typografischen Gepflogenheiten eines europäischen Layouts berücksichtigen}~(Voss 2017, S.~61).
\end{quotation}


\minisec
{Tipps: Schreiben einer wissenschaftlichen Arbeit mit \LaTeX{}}
%
Generell stellt sich für uns das Problem, nun mit \LaTeX{} einfach und rasch eine wissenschaftliche Arbeit zu schreiben.
Den Inhalt nimmt uns \LaTeX{} leider nicht ab, dafür sind wir selbst verantwortlich.
Jedoch können wir uns bei \LaTeX{} auf einige Vorteile verlassen, die wir hier näher betrachten möchten.
Vorweg sei die konsistente Formatierung genannt.

Bevor wir diese Vorteile behandeln, beschäftigen wir uns mit generellen Anforderungen an wissenschaftliche Arbeiten.
Die Erstellung der Gliederung und des Aufbaus einer wissenschaftlichen Arbeit sind meist der erste Schritt für jede Autorin bzw. jeden Autor.
Danach beschäftigt uns der Inhalt, denn jeder dieser Punkte bei der Gliederung will auch mit sinnvollem Inhalt gefüllt sein.
Ausreichendes Datenmaterial (Zahlen, Daten, Fakten) sollte dann gesammelt werden oder sein.
Gute Bilder, ansprechende Diagramme und aussagekräftige Tabellen helfen jeder Leserin und jedem Leser bei der Erfassung des wissenschaftlichen Inhalts.

\minisec{Vorgehensweise: Gliederung und Aufbau}
%
Meist ist der erste Schritt einer wissenschaftlichen Arbeit die Erstellung einer Gliederung der Arbeit (entspricht meist grob dem Inhaltsverzeichnis).
Diese wird dann mit den Betreuern der Arbeit besprochen.
Die folgende Aufzählung zeigt so eine sinnvolle Gliederung einer wissenschaftlichen Arbeit:
%
\begin{itemize}
   \item Titel
   \item Name
   \item Kurzfassung
   \item Inhaltsverzeichnis (bei Bedarf)
\end{itemize}
\begin{enumerate}
   \item Einleitung
      \begin{enumerate}
         \item Problemstellung, Motivation
         \item Vorgehensweise
      \end{enumerate}
   \item Definitionen und Abgrenzungen (Grundlagen, Theorie, Vorarbeiten)
      \begin{enumerate}
         \item Begriff A
         \item Begriff B
         \item \dots
      \end{enumerate}
   \item Hauptteil (eigene Arbeiten inkl. Ergebnisse)
      \begin{enumerate}
         \item Argument 1
         \item Argument 2
         \item \dots
      \end{enumerate}
   \item Zusammenfassung und Schlussfolgerungen (Bewertung und Ausblick)
\end{enumerate}
\begin{itemize}
   \item Literaturverzeichnis
   \item optional Abkürzungsverzeichnis, weitere Verzeichnisse und Anhang
\end{itemize}
%

\minisec{Beispiele zur Typografie}
%
Den Arbeiten \enquote{typokurz -- Einige wichtige typografische Regeln}~(\textcite{Bier:2009}) sowie \citetitle{Struckmann:2007} (\textcite{Struckmann:2007}) entnehmen wir direkt einige Tipps:
%
\begin{enumerate}
   \item \textsc{Auszeichnungen/Hervorhebungen von Text}
      \begin{description}
         \item[Kursive] Eigene Schriftform; \emph{integrierte} 
                      Auszeichnung, die erst auffällt, wenn man an die entsprechende Stelle kommt; 
                      im Normalfall für Auszeichungen im Text am besten geeignet.
         \item[Fette] Normalerweise in Textabschnitten zu vermeiden, viel zu
                      aufdringlich (\emph{aktive} Auszeichnung), zieht direkt die Aufmerksamkeit auf sich (daher für Nachschlagewerke sinnvoll); 
                      für Überschriften, Bezeichnungen von Tabellen und Abbildungen, für Teile von Aufzählungen und Verzeichnissen sowie Tabellenköpfen geeignet;  gelegentlich wird sie auch bei Literaturverweisen im Text verwendet.
         \item[Unterstreichung] Unbedingt zu vermeiden; Überbleibsel aus dem
                      Schreibmaschinenzeitalter, als es nur eine Schriftform auf der Schreibmaschine gab.
         \item[Kapitälchen] Auch nur verwenden, wenn man weiß, was man tut.
                      Das heißt, man (er-)kennt den Unterschied zwischen echten und falschen Kapitälchen.
      \end{description}
   \item \textsc{Striche}
      \begin{description}
         \item[Trennstrich, Bindestrich] wird auch \emph{Divis} genannt und ist
                      ein kurzer Strich. Er dient zur Silbentrennung bzw. zur Verbindung zusammengesetzter Wörter. 
                      In \LaTeX{}: \verb|-|.
         \item[Gedankenstrich] Halbgeviertstrich, länger als der Divis, steht
                      zwischen zwei Leerzeichen (außer in Verbindung mit einem Satzzeichen), \zB{} Ich hoffe sehr -- und das meine ich ganz ehrlich --, Sie bald zu treffen. 
                      In \LaTeX{}: \verb|--|.
         \item[Streckenstrich/Bis-Strich] Halbgeviertstrich ohne Leerzeichen
                      davor und dahinter (Ausnahme: in Verbindung mit Wörtern wird ein Leerzeichen verwendet), \zB{} Linz--Wien, 1--2 Telefonate, 25.9.--28.12., 325 v.Chr. -- 440 n.Chr. 
                      In \LaTeX{}: \verb|--|.
         \item[Auslassungsstrich] Der Halbgeviertstrich dient im Text auch als
                      Auslassungszeichen; in Tabellen sollte dafür ein Geviertstrich (---) verwendet werden, der die Breite von zwei Nullen hat.
                      In \LaTeX{}: \verb|---|.
      \end{description}
   \item \textsc{Absatzformatierung}: Absätze kann man auf zwei Arten
            voneinander trennen: \emph{Einzug} oder \emph{Abstand}. In Bezug auf wissenschaftliche Arbeiten gilt meist: Absätze werden durch einen Einzug von ca. \SI{4}{mm} gekennzeichnet. 
            Abschnitte werden durch einen Abstand von einer Leerzeile gekennzeichnet und im Unterschied zu Absätzen ohne Einzug gesetzt.
%
      \begin{labeling}[\dots]{Flattersatz }
          \item[Flattersatz] Dieser hat den großen Vorteil, dass die 
                      Wortzwischenräume immer gleich groß sind, was positiv für die Lesbarkeit ist. 
                      Andererseits wirkt der Flattersatz eher unruhig, vor allem bei schlechtem Zeilenumbruch.
          \item[Blocksatz] Ob man sich für oder gegen Blocksatz entscheidet, ist
                      abhängig von der Zeilenlänge, der Sprache, in der der Text verfasst wird, dem Mechanismus der Silbentrennung und dem Umbruchalgorithmus der verwendeten Software.
                      Will man für längere Zeilen Blocksatz verwenden, muss man sicherstellen, dass die Software gleichmäßige und enge Wortzwischenräume erzeugt; diese sollten innerhalb einer Zeile gleich groß sein und sich von jenen in der vorangehenden und nachfolgenden Zeile nicht deutlich unterscheiden. 
      \end{labeling}
%
   \item \textsc{Schriften}: Es ist sinnvoll, für längere Texte mit breiten 
            Zeilen eine Schrift mit Serifen und Strichstärkenunterschied zu verwenden.
            Die Serifen (Endstriche) unterstützen einerseits das Auge bei der Zeilenführung und beim Zeilenrücksprung.
            Andererseits führt der Strichstärkenunterschied zu eindeutigeren Wortbildern, was das Lesen sehr erleichtert.
            Am Bildschirm sind serifenlose Schriften bzw. solche ohneStrichstärkenunterschied in der Tat häufig besser zu lesen alsserifenbehaftete Schriften. 
            Daher ist der Vergleich der Schriften auf Papier wichtig.
   \item \textsc{Trennung von Abkürzungen}: Dies ist zu vermeiden. 
            Auch abgekürzte Einheiten sollen nach Möglichkeit nicht von den dazugehörigen Zahlen getrennt werden. 
            Dazu verwenden Sie die Tilde \~{} zwischen Zahl und Einheit (noch besser: die Befehle des \texttt{siunitx}-Pakets).
\end{enumerate}


\minisec{Tipps zum guten Schreiben}

Die folgenden Tipps und Zitate sind direkt dem gut verständlichen Buch von Wolf Schneider~\cite{Schneider:2011} entnommen. 
Sie sind für alle sehr wertvoll, die eine lesbare Arbeit schreiben möchten.
Schneider formuliert dies wie folgt:
%
\begin{quotation}
\emph{
Auf der Basis der korrekten Grammatik muss ich eine Kunst erlernen, die in der Schule ignoriert worden ist: \emph{wie man für Leser schreibt}. \dots 
\newline
Ganz ohne Plage geht das nicht.}
\end{quotation}

Die folgenden Ratschläge gelten für alle Texte, die nicht nur von einer Person gelesen werden sollen. 
Sie stammen direkt aus~\cite{Schneider:2011}:
%
\emph{
\begin{itemize}
   \item Was für alle Texte gilt:
      \begin{itemize}
         \item Einfach draufloszuschreiben ist ein Luxus, den sich keiner
               leisten kann, der von Unbekannten gelesen werden möchte.
               Um seine Leser muss geworben werden, und ganz ohne Plage geht das nicht.
               Am Anfang stehen ein paar schlichte Einsichten -- und der Wille, sich an sie zu halten.
         \item Schon mit den ersten 350 Zeichen kann alles verdorben werden:
               wenn sie nämlich hingehudelt, wenn sie langweilig sind.
         \item Flickwörter müssen gemieden werden, hohle Redensarten ebenso.
               Kürze ist aber \emph{nicht} der oberste Wert: Bilder, Beispiele, Vergleiche sind oft nötig und stets willkommen.
         \item Wenn die Stillehrer sich in einem Punkt einig sind, dann in
               diesem: Der sicherste Weg, die Aufmerksamkeit der Lesenden zu wecken und wachzuhalten, ist der, besonders, bestimmt und konkret zu sein.
      \end{itemize}
   \item Das pralle Wort
         \begin{itemize}
            \item Konkrete, knackige, möglichst kurze Wörter sollten gewählt
                  werden; und in jedem Grenzfall die schlichtesten -- nach Schopenhauers schönem Satz: \emph{Man gebrauche gewöhnliche Worte und sage ungewöhnliche Dinge}.
            \item \emph{Verben} sind die stärksten Wörter, \emph{Adjektive} die
                  schwächsten. 
                  \emph{Synonyme} soll man für die Nebensachen wählen -- für die Hauptsachen nie.
            \item \emph{Anglizismen} gibt es gute und schlechte.
            \item Der größte Feind einer klaren, leserfreundlichen Sprache ist
                  der \emph{akademisch-bürokratische Jargon}.
         \end{itemize}
   \item Die elf Gebote des Satzbaus
         \begin{itemize}
            \item Verboten ist, 
               \begin{enumerate}
                  \item mehr als 6 Wörter zwischen die Teile eines zweiteiligen
                        Verbums oder zwischen Subjekt und Prädikat zu schieben.
                  \item einen Satz länger zu machen, als bei lautem Lesen der
                        Atem reichen würde.
               \end{enumerate}  
               \item Extrem unerwünscht sind 
               \begin{enumerate}
                  \item[3.] eingeschobene Nebensätze.
                  \item[4.] vorangestellte Attribute.
               \end{enumerate}
               \item Unerwünscht sind 
                \begin{enumerate}
                   \item[5.] hartnäckig wiederkehrende Satzanfänge.
                   \item[6.] atemlos gesetzte Punkte.
                \end{enumerate}
               \item \textbf{Erwünscht} sind 
                \begin{enumerate}
                   \item[7.] Hauptsätze -- immer die erste Wahl.
                   \item[8.] oft: angehängte Nebensätze -- ohne Handlung und mit Augenmaß.
                   \item[9.] manchmal: vorangestellte Nebensätze, wenn sie kurz sind.
                   \item[10.] Sätze wie Pfeile, vorwärtsstrebend.
                   \item[11.] mind. 5 der 7~Satzzeichen in \emph{jedem} Text. \newline
                              Diese 7 sind: Punkt, Komma, Doppelpunkt, Fragezeichen, Ausrufezeichen, Semikolon und Gedankenstrich.
                \end{enumerate}
         \end{itemize}
\end{itemize}
}
%
Bei Diplom- und Doktorarbeiten kann ein bisserl von obigen abgewichen werden.
So fängt eine dieser Arbeiten normalerweise mit einer Übersicht über Ihr Vorhaben sowie jenen Fragestellungen an, die Sie im Lauf der nächsten 70 bis 200~Seiten beantworten möchten.


% Hier wird das Literaturverzeichnis eingefügt.
\printbibliography

\end{document}