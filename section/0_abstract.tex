Transient electromagnetic (TEM) soundings can provide valuable insights,
into the shallow subsurface, which makes this geophysical method popular for 
environmental appliations like groundwater exploration and monitoring saline water intrusions.
To properly interpret TEM data, inversion algorithms are used and 
current deterministic approaches utilise a regularisation parameter (lambda), 
which balances model-data fit with the complexity of the model.
In this thesis, I introduce the L-Curve method, developed to solve the search for an optimal lambda, 
to the inversion of TEM data, while testing three different search algorithms for 
the automatic interpretation of the L-Curve.
In particular, I test the L-Curve method an TEM data collected at the 
Martenhofer Lacke in the \textit{Nationalpark Neusiedlersee - Seewinkel}.
For this data set the ``Gradient Based'' search algorithm performed best compared to the other two automatic search methods by working
for $59.16\,\%$ of the tested cases, yet we found that the visual interpretation of the L-curve was the most reliable technique to find an optimal
lambda, working for $73.96\,\%$ of the soundings.
Using the obtained optimal lambda values, 
we were able to resolve the shallowest three layers with a single square loop with the side lenght of $12.5\,\text{m}$ 
and a current of $4.1\,\text{A}$: 
An aquifer between the depths of $4-10.5\,\text{m}$ with a resistivity of $30\,\Omega\text{m}$,
confined by two clay layers with the resistivities of $10-15\,\Omega\text{m}$.