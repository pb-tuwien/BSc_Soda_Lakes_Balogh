The raw data were plotted in terms of the measured secondary magnetic field ($\frac{\text{V}}{\text{m}^2}$), referred to as ``impulse response'' in this thesis, 
and the conversion to apparent resistivity $\rho_{\text{a}}$ ($\Omega\text{m}$), using \cref{eq:apparent_resistivity}.
The latter reveal larger contrasts and permit an easier visual identification of outliers.
The data gathered in the two surveys was processed separately.

\subsection{Filtering of the Data}
\label{subsec:filtering}
After plotting, the soundings of the first survey (22nd May 2024) were split into two groups:
``Clean'' and ``noisy'' data, with noisy referring to sounding close to the road 
(``M014'', ``M043'', and ``M044'' as seen in \cref{fig:may-map}).
The noisy soundings are likely related to measurements affected by interferences of possible
anthropogenic structures and parasitic electromagnetic fields contminating the data~\parencite{christiansenTransientElectromagneticMethod2006}.
\cref{fig:good-signal-may} shows the clean soundings of the first survey (22nd May 2024) with the $12.5\times12.5\,\text{m}$ loop 
and \cref{fig:err-signal-may} shows the noisy data.
In \cref{fig:good-signal-may}(b) it can be clearly seen that the effects of the turn-off ramp 
as decribed by \textcite{aignerFlexibleSingleLoop2021} influence the measurents until about $8\,\mu\text{s}$, 
which is consistent with the $4.2-8.6\,\mu\text{s}$, found by \textcite{aignerFlexibleSingleLoop2021} for conductive media.
At about $240\,\mu\text{s}$ \cref{fig:good-signal-may}(b) shows how the various curves start having distortions
and \cref{fig:good-signal-may}(a) shows that measured impulse responses are becoming a similar order of magnitude as the measured errors.
For this reason the measurements were filtered by excluding all time gates outside of the time range of $8-210\,\mu\text{s}$ 
as shown in \cref{fig:good-signal-may}(c) and (d).
\cref{fig:good-signal-may}(d) shows the pattern, that the $\rho_{\text{a}}$ values first decrease for most soundings 
and start increasing for all soundings for a time larger than $30\,\mu\text{s}$.

\begin{figure*}[!ht]
    \centering
    \includegraphics[width=0.8\linewidth]{data/20240522/TEM-data/04-first_look/20240522_good_soundings.png}
    \caption[Response Curves of the May Survey]{
        Raw and filtered impulse responses and apparent resistivities from the 22nd May 2024 survey, where anomalies are excluded.
        The first sounding (M001) is colored dark purple, the last sounding (M045) is yellow, 
        and all soundings in between are assigned a color from a uniform distribution between the two border colors.
        Subfigure (a) shows the raw impuls response as well as the measured errors as dashed lines, 
        (b) shows the computed apparent resistivities, 
        (c) show the filtered impulse response, the measured error as dashed lines, and greyed out raw data as comparison,
        (d) shows the resulting filtered apparent resistivities and greyed out raw values.
    }
    \label{fig:good-signal-may}
\end{figure*}

The apparent resistivities of the noisy soundings as seen in \cref{fig:err-signal-may}(b) 
show no clear pattern due to high distortions in the curves.
The anomalous measurements shown in \cref{fig:err-signal-may} were still processed, to test how well they compare to the other soundings.
In order to make it more comparable the same time range of $8-210\,\mu\text{s}$ was used for the filtering (\cref{fig:err-signal-may}(d)).

\begin{figure*}[!ht]
    \centering
    \includegraphics[width=0.8\linewidth]{data/20240522/TEM-data/04-first_look/20240522_err_soundings.png}
    \caption[Anomalous Responses of the May Survey]{
        The anomalous raw and filtered impulse responses and apparent resistivities from the 22nd May 2024 survey.
        Subfigure (a) shows the raw impuls response as well as the measured errors as dashed lines, 
        (b) shows the computed apparent resistivities, 
        (c) show the filtered impulse response, the measured error as dashed lines, and greyed out raw data as comparison,
        (d) shows the resulting filtered apparent resistivities and greyed out raw values.
    }
    \label{fig:err-signal-may}
\end{figure*}

The clean soundings (\cref{fig:good-signal-oct}) and anomalies (\cref{fig:err-signal-oct}) of the second survey (8th October 2024) 
with the $6.25\times6.25\,\text{m}$ loop were processed in a similar way.
Comparing \cref{fig:good-signal-oct}(b) with \cref{fig:good-signal-may}(b) shows that the effects of the turn-off ramp both end at $8\,\mu\text{s}$, 
while the distortions in the data from the second survey start earlier at $100-110\,\mu\text{s}$.
The impulse response also becomes the same order of magnitude as the measured error~(\cref{fig:good-signal-oct}(a)) at the same time~($110\,\mu\text{s}$).
Thus all the time gates outside the time range $8-110\,\mu\text{s}$ were filteredas, shown in \cref{fig:good-signal-oct}(c) and (d).
\cref{fig:good-signal-oct}(d) shows that in contrast to the soundings with the larger loop (\cref{fig:good-signal-may}(d)), 
the $\rho_\text{a}$ values do not decrease in the early times, but also start increasing after $30\,\mu\text{s}$.

\begin{figure*}[!ht]
    \centering
    \includegraphics[width=0.8\linewidth]{data/20241008/TEM-data/04-first_look/20241008_good_soundings.png}
    \caption[Response Curves of the October Survey]{
        Raw and filtered impulse responses and apparent resistivities from the 8th October 2024 survey, where anomalies are excluded.
        The first sounding (M001) is colored dark purple, the last sounding (M066) is yellow, 
        and all soundings in between are assigned a color from a uniform distribution between the two border colors.
        Subfigure (a) shows the raw impuls response as well as the measured errors as dashed lines, 
        (b) shows the computed apparent resistivities, 
        (c) show the filtered impulse response, the measured error as dashed lines, and greyed out raw data as comparison,
        (d) shows the resulting filtered apparent resistivities and greyed out raw values.
        }
        \label{fig:good-signal-oct}
    \end{figure*}
    
The noisy soundings shown in \cref{fig:err-signal-oct} can be divided into two groups:
Measurements close to man-made structures, such as the road (``M002'', ``M009'', ``M010'', and ``M024'' as seen in \cref{fig:may-map}), 
which show similarities to the noisy soundings of the first survey (\cref{fig:err-signal-may}(b)).
``M002'', ``M009'', and ``M010'' show significantly lower apparent resistivity values, than the other soundings --
at $40\,\mu\text{s}$ $10\,\Omega\text{m}$~(\cref{fig:err-signal-oct}) instead of $20\,\Omega\text{m}$~(\cref{fig:good-signal-oct}).
This could be due to coupling effects, as the impulse response curves (\cref{fig:err-signal-oct}(a)) look similar to the 
example for capacitive coupling described by \textcite{christiansenTransientElectromagneticMethod2006}.
``M024'' shows higher a apparent resistivity of $30\,\Omega\text{m}$ at $40\,\mu\text{s}$ and a much steeper increase in
the $\rho_\text{a}$ values~(\cref{fig:err-signal-oct})

Measurements within the lake (``M057''-``M065'', except for ``M062'', as seen in \cref{fig:may-map}), 
which show effects of the turn-off ramp until $12\,\mu\text{s}$ (\cref{fig:err-signal-oct}(b)).
For these sounding the loop was submerged in the lake (up to $30\,\text{cm}$ deep), 
where the water is a good conductor due to its high salinity with resistivities of $0.78-7.16\,\Omega\text{m}$~\parencite{borosHowCanUnique2025}.
This also breaks the assumption of a layered half-space, 
which is necessary for the inversion of TEM data~\parencite{christiansenTransientElectromagneticMethod2006}, and 
thus all soundings within the lake (marked orange in \cref{fig:oct-map}) were excluded from further processing.

\begin{figure*}[!ht]
    \centering
    \includegraphics[width=0.8\linewidth]{data/20241008/TEM-data/04-first_look/20241008_err_soundings.png}
    \caption[Anomalous Responses of the October Survey]{
        The anomalous raw and filtered impulse responses and apparent resistivities from the 8th October 2024 survey.
        Subfigure (a) shows the raw impuls response as well as the measured errors as dashed lines, 
        (b) shows the computed apparent resistivities, 
        (c) show the filtered impulse response, the measured error as dashed lines, and greyed out raw data as comparison,
        (d) shows the resulting filtered apparent resistivities and greyed out raw values.
    }
    \label{fig:err-signal-oct}
\end{figure*}

\FloatBarrier
\subsection{Inversion Parameters}
\label{subsec:inversion-parameters}

In this section we evaluate the effect in the inversion due to varying parameters, 
which are necessary for the algorithm by \textcite{aignerFlexibleSingleLoop2021}:
The relative error vector, range of $\lambda$ values, and the layer distribution.
The relative error vector specifies the corresponding relative error for each recorded impulse response of a time gate.
It can be derived from the absolute errors, which the device records automatically or set manually.
\textcite{aignerSensitivityAnalysisInverted2024} conducted a survey at the same location with a $50.0\times50.0\,\text{m}$ loop
and found that the device overestimates the error.
Thus \textcite{aignerSensitivityAnalysisInverted2024} used a constant $1.5\,\%$ relative error vector for the inversion.
For the data collected in both surveys of this thesis, the inversions were run defining an relative error vector of $1.5\,\%$.

The inversion algorithm of \textcite{aignerFlexibleSingleLoop2021} also requires a regularisation parameter $\lambda$.
This thesis focusses on finding an optimal $\lambda$ by using the L-Curve method.
The L-Curve method was implemented in a way, that it requires an interval of two $\lambda$ values,
which are used to create a range $20$ of logarithmically spaced $\lambda$ values.
These are then used to compute $20$ inversions of which the corresponding RMS misfit 
and model roughness is plotted to create the L-Curve.
The study from \textcite{aignerSensitivityAnalysisInverted2024} uses $\lambda$ values between $40.5-50$ for the inversion.
To allow divergence from these values, a lower bound of $5$ and an upper bound of $100$ was chosen for this thesis.

The layer distribution specifies the number and thicknesses of the layers characterising
the subsurface model, which is used for the inversion.
\textcite{welkensComparisonSubsurfaceGeophysical2025} shows that a layer model with $1\,\text{m}$-Layers until $5\,\text{m}$ depth 
and $1.5\,\text{m}$-Layers below that, until a maximal depth of $10\,\text{m}$ optimises the inversion
speed and model fit for data gathered with a $6.25\times6.25\,\text{m}$ loop and $4.1\,\text{A}$ of current.
For a first survey of this thesis a $12.5\times12.5\,\text{m}$ loop was 
and using \cref{fig:inv-depth-10m}(c) shows that the inversion algorithm struggles to fit the late time data, 
when using the layer distribution suggested by \textcite{welkensComparisonSubsurfaceGeophysical2025}.
\cref{fig:inv-depth-20m}(c) shows that this issue can be resolved by choosing a maximal depth of $20\,\text{m}$ 
and keeping the other layer thickness suggestions from \textcite{welkensComparisonSubsurfaceGeophysical2025}.
The visual differences between \cref{fig:inv-depth-10m}(c) and \cref{fig:inv-depth-20m}(c) are minimal, 
and thus the relative root-mean-square (RMS) of the misfit between modelled and observed data was used to quantify the imporvement,
which decreased from $2.32\,\%$ to $2.09\,\%$ using $20\,\text{m}$ instead of $10\,\text{m}$ as the maximal depth.

\begin{figure*}[ht]
    \centering
    \includegraphics[width=0.8\linewidth]{data/parameter_testing/M028_max_depth_10m.png}
    \caption[Inversion With 10m Maximum Depth]{
        Inversion results for a representative sounding (M028) for the first survey (22nd May 2024)
        with the layer distribution of $1\,\text{m}$ thicknesses until $5\,\text{m}$ depth and
        $1.5\,\text{m}$ until $10\,\text{m}$ depth.
        Subfigure (a) shows the final subsurface model of the resistivity, 
        (b) shows the comparison between the modelled and measured impulse response as well as the measured error as a dashed line,
        (c) shows the comparison between the modelled and measured apparent resistivities.
    }
    \label{fig:inv-depth-10m}
\end{figure*}

\begin{figure*}[ht]
    \centering
    \includegraphics[width=0.8\linewidth]{data/parameter_testing/M028_max_depth_20m.png}
    \caption[Inversion With 20m Maximum Depth]{
        Inversion results for a representative sounding (M028) for the first survey (22nd May 2024)
        with the layer distribution of $1\,\text{m}$ thicknesses until $5\,\text{m}$ depth and
        $1.5\,\text{m}$ until $20\,\text{m}$ depth.
        Subfigure (a) shows the final subsurface model of the resistivity, 
        (b) shows the comparison between the modelled and measured impulse response as well as the measured error as a dashed line,
        (c) shows the comparison between the modelled and measured apparent resistivities.
    }
    \label{fig:inv-depth-20m}
\end{figure*}

\FloatBarrier
\subsection{The L-Curve Method}
\label{subsec:manual-l-curve}

The L-Curve technique can be used to plot the model roughness against the model misfit for varying regularisation parameter ($\lambda$) values, 
which can be used to find the optimal lambda $\lambda_\text{opt}$~\parencite{hansenAnalysisDiscreteIllPosed1992}.
This can be done by visually finding the ``corner'' of the L-curve, 
which were computed with the parameters from \cref{subsec:inversion-parameters}:
$1.5\,\%$ as the relative error, $20$ logarithmically spaced $\lambda$ values in the range $5-100$, 
and the layer distribution from \textcite{welkensComparisonSubsurfaceGeophysical2025} but with a maximal depth of $20\,\text{m}$.
For the first survey the L-Curves were computed for the clean as well as noisy soundings, 
where we found no significant visual difference between the clean and noisy data, and thus
excluded all noisy data from further analysis.
The remaining $43$ soundings of the first survey were manually analysed 
and we were able to find an optimal lambda $\lambda_\text{opt}$ for $83.33\,\%$ of the soundings.
Doing the same for the $54$ clean soundings of the second survey resulted in finding a $\lambda_\text{opt}$ for $66.67\,\%$
of the soundings.

\cref{fig:lcurve-ideal,fig:lcurve-usable} show two L-Curves, which allowed finding an $\lambda_\text{opt}$.
\cref{fig:lcurve-ideal} shows the L-Curve for the sounding ``M028'' of the first survey 
and illustrates the expected L-Curve as it is described by \textcite{hansenAnalysisDiscreteIllPosed1992}.
Based on \cref{fig:lcurve-ideal} we determined the optimal lambda for this sounding to be at $13$,
which minimises both the RMS misfit and the model roughness.
\cref{fig:lcurve-usable} shows the L-Curve for the sounding ``M003'' of the second survey 
and was selected to showcase how the computed L-Curve can deviate from the expected ideal, 
but still provide usefull information.
\cref{fig:lcurve-usable} shows contrary to what is expected, 
the RMS misfit increases between lambdas $8-5$ and decreases between $63-100$, 
but the curve still has the characteristic ``corner'' at the $\lambda$ value $8$,
which we determined to be the optimal lambda.

\begin{figure*}[ht]
    \centering
    \includegraphics[width=0.55\linewidth]{data/20240522/TEM-data/07-inversion_analysis/l_curve_M028.png}
    \caption[Example of an ideal L-Curve]{
        L-Curve of the sounding M028 in the first survey (22nd May 2024), 
        which shows the expected characteristics of a L-Curve.
    }
    \label{fig:lcurve-ideal}
\end{figure*}

\begin{figure*}[ht]
    \centering
    \includegraphics[width=0.55\linewidth]{data/20241008/TEM-data/07-inversion_analysis/l_curve_M003.png}
    \caption[Example of an usable L-Curve]{
        L-Curve of the sounding M003 in the second survey (8th October 2024), 
        which shows a L-Curve that diverges from the expected, 
        but still can be used to find an optimal lambda.
    }
    \label{fig:lcurve-usable}
\end{figure*}

\cref{fig:lcurve-inverted,fig:lcurve-bad} show two examples of the L-Curve method not working to find an
optimal lambda for TEM soundings.
Generally for an increasing $\lambda$ a descreasing model roughness 
and an increasing model misfit is expected~\parencite{hansenAnalysisDiscreteIllPosed1992}, 
but \cref{fig:lcurve-inverted}, which was caclulated for the sounding ``M045'' of the first survey, 
shows a decreasing model misfit (RMS misfit) for an increasing $\lambda$.
It is plausible that this behaviour changes with further increases in the $\lambda$ values, 
as it happens in \cref{fig:lcurve-usable} between $8-53$.
\cref{fig:lcurve-bad} shows an example, which was computed for the sounding ``M019'' of the second survey, 
where the L-Curve method produced no cohesive curve, but rather collection of seemingly randomly positioned points.

\begin{figure*}[ht]
    \centering
    \includegraphics[width=0.55\linewidth]{data/20240522/TEM-data/07-inversion_analysis/l_curve_M045.png}
    \caption[Example of an inverted L-Curve]{
        L-Curve of the sounding M045 in the first survey (22nd May 2024), 
        which shows an increasing RMS with an increasing roughness, 
        which is the opposite to the expected characteristics of a L-Curve.
    }
    \label{fig:lcurve-inverted}
\end{figure*}

\begin{figure*}[ht]
    \centering
    \includegraphics[width=0.55\linewidth]{data/20241008/TEM-data/07-inversion_analysis/l_curve_M019.png}
    \caption[Example of a bad L-Curve]{
        ``L-Curve'' of the sounding M019 in the second survey (8th October 2024), 
        which shows a seemingly random collection of points instead of a L-Curve.
    }
    \label{fig:lcurve-bad}
\end{figure*}

We found that, modes of the $\lambda$ values for the soundings of the two surveys where $13$ (first) and $11$ (second).
These values are significantly smaller than the $\lambda$ of $50$ used by \textcite{aignerSensitivityAnalysisInverted2024} for the same location,
but with a different loop size and using a diffent inversion algorithm, 
which is based on the blocky instead of the smooth inversion from \texttt{PyGIMLi}~\parencite{ruckerPyGIMLiOpensourceLibrary2017} 
and decreases the $\lambda$ after each iteration of the inversion by a ``cooling factor''.

\FloatBarrier
\subsection{Finding an Optimal Lambda}
\label{subsec:optimal-lambda}

The clean soundings of both surveys were used to test the three search algorithms presented in \cref{subsubsec:inversion}:
All algorithms try finding the $\lambda$ corresponding to the point on the L-CUrve with the maximal curvature.
The ``cubic-spline'' based approach fits a cubic spline function, 
which is then used to compute the curvatures of each point~\parencite{lloydUseLcurveMethod1997}.
The ``gradient-based'' method utilises the \texttt{gradient()} function from the \texttt{numpy} library to achieve the same.
The ``golden section search'' is an iterative search algorithm based on \textcite{cultreraSimpleAlgorithmFind2020}.

In early versions all three search algorithms performed poorly 
and we attributed this to the fact that the RMS misfit and the model roughness are $5-6$ orders of magnitude apart,
which can be seen when looking at the axis of \cref{fig:lcurve-usable}.
\cref{fig:example-norm-false} shows how all three search algorithms return a $\lambda$ between $41-62$,
which is higher than the visually determined $\lambda_\text{opt}$ at $13$.
To counteract this a linear normalisation, using the formula
$\mathbf{a}_i = \frac{\mathbf{a}_\text{i}-\mathbf{a}_\text{min}}{\mathbf{a}_\text{max}-\mathbf{a}_\text{min}}$, was 
applied to both RMS and roughness.
\cref{fig:example-norm-true} shows how with the normalisation all three search algorithms return a value ($12-15$) around 
the visual optimum of $13$. Hence all further searches for an optimal lambda were carried out with the normalisation.

\begin{figure*}[ht]
    \centering
    \includegraphics[width=0.55\linewidth]{data/parameter_testing/M028_normalization_false.png}
    \caption[Lambda Search Without Normalisation]{
        Automatically searching for an optimal lambda of the sounding M028 in the first survey (22nd May 2024) 
        without applying a normalisation to the L-Curve.
    }
    \label{fig:example-norm-false}
\end{figure*}

\begin{figure*}[ht]
    \centering
    \includegraphics[width=0.55\linewidth]{data/parameter_testing/M028_normalization_true.png}
    \caption[Lambda Search With Normalisation]{
        Automatically searching for an optimal lambda of the sounding M028 in the first survey (22nd May 2024) 
        while applying a normalisation to the L-Curve.
    }
    \label{fig:example-norm-true}
\end{figure*}

\cref{fig:example-norm-true} demontrates that all three search algorithms are able to find the optimal lambda for a 
L-Curve, which resembles an ideal L-Curve as described by \textcite{hansenAnalysisDiscreteIllPosed1992}.
\cref{fig:search-working} shows that all three search algorithms find the optimal lambda at $8$ when there is a clear
point with a maximal curvature, even if the L-Curve deviates from the expected ideal.
\cref{fig:search-not-working} demonstrated how the various search algorithm find different ``optimal'' lambda values, 
when the curvature of the computed L-Curve has no distinct maximum, 
which is exemplified by sounding ``M020'' of the first survey.
When inspecting the L-Curve shown in \cref{fig:search-not-working}, we determined $\lambda_\text{opt}$ to be $15$:
The ``cubic-spline'' and ``gradient'' based search algorithms overestimated $\lambda_\text{opt}$ to be between $45-53$,
while the ``golden section search'' underestimated $\lambda_\text{opt}$ to be $7$.

\begin{figure*}[ht]
    \centering
    \includegraphics[width=0.55\linewidth]{data/20240522/TEM-data/07-inversion_analysis/comparison_M020.png}
    \caption[Example Lambda Search Working With Non-Ideal L-Curve]{
        Automatically searching for an optimal lambda of the sounding M020 in the first survey (22nd May 2024), 
        which showcases how performance the different search algorithms deviate.
    }
    \label{fig:search-not-working}
\end{figure*}

\begin{figure*}[ht]
    \centering
    \includegraphics[width=0.55\linewidth]{data/20241008/TEM-data/07-inversion_analysis/comparison_M003.png}
    \caption[Example Lambda Search Not Working]{
        Automatically searching for an optimal lambda of the sounding M003 in the second survey (8th October 2024), 
        which showcases how the different search algorithms perform well, even when the computed L-Curve deviates from the theoretical ideal,
        but some expected characteristic of an L-curve can be found.
    }
    \label{fig:search-working}
\end{figure*}

These examples highlight the strenghts and shortcomings of each search algorithm. 
\cref{tab:search-summary} shows a summary of how the algorithms performed overall 
and comparing this to a visual inspection of the generated L-Curve.
The L-Curve method worked for $83.33\,\%$ of the soundings with the first survey and 
only for $66,67\,\%$ of the soundings with the second survey.
\cref{tab:search-summary} shows that this trend continues for the performance of the
search algorithms.
We evaluated each search algorithm by comparing the obtained $\lambda$ with our visual analysis
from \cref{subsec:manual-l-curve}.

We classified an automatic search ``successful'', 
if the returned $\lambda$ was within an interval of $15\,\%$ around the $\lambda_\text{opt}$ 
from our manual inspection.
\cref{tab:search-summary} shows that for the first survey the ``Golden Section Search'' was more successful at finding
the $\lambda$ with $74.29\,\%$ than the ``Gradient Based'' approach with $71.43\,\%$, 
while for the second survey the Gradient Based method worked better with $47.22\,\%$ 
than the Golden Section Search with $41.67\,\%$.
Throughout both surveys the ``Cubic Spline'' based search performed the worst with
$60.00\,\%$ for the first and $33.33\,\%$ for the second survey.

We also computed the means of the relative deviations between the visually obtained $\lambda$ 
and the automatically computed value for each search algorithm.
We introduced this metric beside the binary ``Successrate'' to quantify the reliability of each search algorithm:
By measuring the mean deviations, we not only see if the $\lambda_\text{opt}$ was found, 
but also how far off the lambda searches were on average.
\cref{tab:search-summary} shows that the Gradien Based approach had the lowest
mean relative deviations with $15.89\,\%$ (first survey) and $39.57\,\%$ (second survey).
The Golden Section Search deviated the second least for the first survey with $17.55\,\%$, 
but performed the worst for the second survey with $118.28\,\%$.
The Cubic Spline method showed a mean relative deviation of $32.66\,\%$ for the first survey 
and $65.21\,\%$ for the second.

\cref{tab:search-summary} also shows the modes of the found $\lambda_\text{opt}$.
For the first survey the Gradient Based search slightly overestimates the lambda with $15$ compared to 
the mode of visually determined values with $13$,
while the Golden Section Search underestimates it with $12$.
In both cases the mode of the lambda values could be used as an approximation 
of the optimal lambda value, where the L-Curve method did not work 
or the search algorithm was not able to find $\lambda_\text{opt}$.
For the second survey the Golden Section Search also slighly underestimates the lambda value with $9$
compared to the mode of the visually determined lambda values with $11$ (see \cref{tab:search-summary}), 
while the Gradient Based approach and Cubic Spline method both return even lower lambda values with $7$ and $5$ respectively.

\begin{table}[!ht]
    \centering
    \caption[Summary of the search for an optimal Lambda]{
        Summary of the search for an optimal lambda for each sounding, comparing the visual identification 
        with the various automated search algorithms.
        }
    \resizebox{\columnwidth}{!}{
        \begin{tabular}{lccc}
            \toprule
            \textbf{Search Type} & \textbf{Successrate (\%)} & \textbf{Mode of Lambdas} & \textbf{Mean relative deviation (\%)}\\
            \midrule
            \textbf{22nd May 2024} \\
            Visual & 83.33 & 13 & - \\
            Cubic Spline & 60.00 & 28 & 32.66 \\
            Gradient Based & 71.43 & 15 & 15.89 \\
            Golden Section & 74.29 & 12 & 17.55 \\
            \textbf{8th October 2024} \\
            Visual & 66.67 & 11 & - \\
            Cubic Spline & 33.33 & 7 & 65.21 \\
            Gradient Based & 47.22 & 5 & 39.57 \\
            Golden Section & 41.67 & 9 & 118.28 \\
            \bottomrule
        \end{tabular}
    }
    \label{tab:search-summary}
\end{table}

All comparison plots can be found at \url{https://github.com/pb-tuwien/BSc_Soda_Lakes_Balogh.git}
under "data/2024xxxx/TEM\-data/07\-inversion\_analysis/comparison\_M0yy.png", where xxxx is either 
"0522" for the first or "1008" for the second survey and yy being the sounding number 
(either 01 to 45 or 01 to 66 -- exluding the numbers for the noisy soundings).
A consistent trend throughout our analysis of the search algorithms (\cref{tab:search-summary})
is that the first survey performed better for the L-Curve method, than the second survey.

\FloatBarrier
\subsection{Comparing Inversion Results}\label{subsec:inversion-results}

In this section we compare the inversion results computed for the different acquisition parameters 
(which were described in \cref{tab:sounding-parameters}).
For this we ran the inversions for all soundings, 
where we were able to find an $\lambda_\text{opt}$ using the L-Curve method.
Ideally this comparison would be done at the same time as well as the same location.
For this study the data was gathered in two survey, which were $4$ months apart.
We identified three different configurations:
Measurements with the loop size $12.5\times12.5\,\text{m}$ and a current of $4.1\,\text{A}$,
which we reference as ``large loop and large current''.
Soundings with the $12.5\times12.5\,\text{m}$ 
and a current of $1.0\,\text{A}$ (``large loop and small current'').
Measurements with the $6.25\times6.25\,\text{m}$ loop and a current of $4.1\,\text{A}$,
which we call ``small loop and large current''.
The different ``time keys'' used in the first survey (as seen in \cref{tab:sounding-parameters})
were not considered, because the late time data was filtered due to a high noise level.

For the large loop and small current we chose sounding ``M028'' (first survey) 
and for the small loop and large current we chose sounding ``M052'' (second survey), 
because these two soundings were done at the same location (as seen in \cref{fig:may-map,fig:oct-map}).
For the large loop and large current we have no data at this location, 
so we chose closest sounding to the location, which was ``M002'' (first survey).

The inversion results for ``M028'' are shown in \cref{fig:inv-big-small} 
and for ``M052'' in \cref{fig:inv-small-big}.
\cref{fig:inv-big-small}(c) shows how with the large loop and small current a three-layer model is fitted: 
A $4.0\,\text{m}$-layer with $17.5\,\Omega\text{m}$, a second $9.5\,\text{m}$-layer with $13.5\,\Omega\text{m}$, 
and the start of a third layer with $20.0\,\Omega\text{m}$.
The model seems to fit the data adequately as can be seen in \cref{fig:inv-big-small}(a) and (b), 
which is also supported by a relative RMS misfit of $2.09\,\%$.

\begin{figure*}[ht]
    \centering
    \includegraphics[width=\linewidth]{data/location_comparison/M028_same_location_12.5.png}
    \caption[Inversion of Sounding with the Large Loop and Small Current]{
        Optimised inversion of the sounding M028 of the first survey, which was measured with a $12.5\times12.5\,\text{m}$ loop
        and $1.0\,\text{A}$ of current.
        This sounding was done at the same location as M052 of the second survey.
        Subfigure (a) shows the comparison between modelled and measured impulse response as well as the measured error,
        (b) shows the comparison between modelled and measured apparent resistivities,
        (c) shows the final subsurface model of the resistivity,
        (d) shows the L-curve of this sounding with the chosen lambda value marked in orange.
        }
        \label{fig:inv-big-small}
    \end{figure*}
    
For the same location, four moths later the small loop with the large current produces a model with four
layers layers, as seen in \cref{fig:inv-small-big}(c): 
A $4.0\,\text{m}$-layer with $16.5\,\Omega\text{m}$, a second $6.5\,\text{m}$-layer with $18.0\,\Omega\text{m}$,
a third $6.0\,\text{m}$-layer with $16.5\,\Omega\text{m}$ and the start of a fourth layer with $38.0\,\Omega\text{m}$.
\cref{fig:inv-small-big}(b) shows that after $40\,\mu\text{s}$ some distortions in the observed data could not be fitted,
which resulted in an relative RMS misfit of $2.23\,\%$.

\begin{figure*}[ht]
    \centering
    \includegraphics[width=\linewidth]{data/location_comparison/M052_same_location_6.25.png}
    \caption[Inversion of Sounding with the Small Loop and Large Current]{
        Optimised inversion of the sounding M052 of the second survey, which was measured with a $6.25\times6.25\,\text{m}$ loop
        and $4.1\,\text{A}$ of current.
        This sounding was done at the same location as M028 of the first survey.
        Subfigure (a) shows the comparison between modelled and measured impulse response as well as the measured error,
        (b) shows the comparison between modelled and measured apparent resistivities,
        (c) shows the final subsurface model of the resistivity,
        (d) shows the L-curve of this sounding with the chosen lambda value marked in orange.
    }
    \label{fig:inv-small-big}
\end{figure*}

\cref{fig:inv-big-big} represents the measurements collected with the large loop and large current
and in (c) similar four layers can be seen as in \cref{fig:inv-small-big}(c) but with a more distinct second layer:
A $4.0\,\text{m}$-layer with $18.5\,\Omega\text{m}$, a second $6.5\,\text{m}$-layer with $31.0\,\Omega\text{m}$,
a third $6.0\,\text{m}$-layer with $7.5\,\Omega\text{m}$ and the start of a fourth layer with $39.0\,\Omega\text{m}$.
\cref{fig:inv-big-big}(b) shows that the modelled apparent resistivity curve has its mininum at $20\,\mu\text{s}$,
while the observed $\rho_\text{a}$ curve has its minimun between $30-40\,\mu\text{s}$.
The inversion algorithm also was not able to fit the last time gate of the observed data (seen in \cref{fig:inv-big-big}(b)),
which results in the highest relative RMS misfit of the sounding discussed in this section with $3.21\,\%$

\begin{figure*}[ht]
    \centering
    \includegraphics[width=\linewidth]{data/location_comparison/M002_similar_location_12.5.png}
    \caption[Inversion of Sounding with the Large Loop and Large Current]{
        Optimised inversion of the sounding M002 of the first survey, which was measured with a $12.5\times12.5\,\text{m}$ loop
        and $4.1\,\text{A}$ of current.
        Subfigure (a) shows the comparison between modelled and measured impulse response as well as the measured error,
        (b) shows the comparison between modelled and measured apparent resistivities,
        (c) shows the final subsurface model of the resistivity,
        (d) shows the L-curve of this sounding with the chosen lambda value marked in orange.
    }
    \label{fig:inv-big-big}
\end{figure*}

\cref{fig:inv-small-big-good} shows the inversion results for ``M050'' of the second survey (small loop and big current),
which we included because it resolves the same high resistivity second layer as the large loop with large current configuration
(\cref{fig:inv-big-big}).
\cref{fig:inv-small-big-good}(c) shows the following four layer model:
A $4.0\,\text{m}$-layer with $14.0\,\Omega\text{m}$, a second $6.5\,\text{m}$-layer with $25.5\,\Omega\text{m}$,
a third $6.0\,\text{m}$-layer with $10.0\,\Omega\text{m}$ and the start of a fourth layer with $42.5\,\Omega\text{m}$.
The fit is adequate, which can be seen in \cref{fig:inv-small-big-good}(b), and thus results in a relative RMS misfit 
of $2.06\,\%$.

\begin{figure*}[ht]
    \centering
    \includegraphics[width=\linewidth]{data/location_comparison/M050_good_result_6.25.png}
    \caption[Good Result with the Small Loop and Large Current]{
        Optimised inversion of the sounding M050 of the second survey, which was measured with a $6.25\times6.25\,\text{m}$ loop
        and $4.1\,\text{A}$ of current.
        Subfigure (a) shows the comparison between modelled and measured impulse response as well as the measured error,
        (b) shows the comparison between modelled and measured apparent resistivities,
        (c) shows the final subsurface model of the resistivity,
        (d) shows the L-curve of this sounding with the chosen lambda value marked in orange.
    }
    \label{fig:inv-small-big-good}
\end{figure*}

The models obtained for ``M002'' (first survey) and ``M050'' (second survey) show the same four layers with similar resistivity values,
which is summarised in \cref{tab:inversion-model}.
\textcite{aignerStochasticInversionTransient2025} collected TEM data with a $12.5\times12.5\,\text{m}$ loop and 
$4\,\text{A}$ of current along a profile north of the Martenhofer Lacke.
This resembles the acquisition parameters from the soundings ``M001'' and ``M002'' from the first survey in this thesis 
(the later is shown in \cref{fig:inv-big-big}).
\textcite{aignerStochasticInversionTransient2025} inverted the data using stochastic methods and found the same
first three layers described in \cref{tab:inversion-model}, but did not find a resistive fourth layer.
\cref{tab:inversion-model} shows a conductive first layer with $14-18\,\Omega\text{m}$, which could indicate saturated
clay layer (\cref{tab:resistivity}), a more resistive second layer with $25-31\,\Omega\text{m}$, 
which could be a saturated sand layer with high salinity (first aquifer), and a third conductive layer with $7-10\,\Omega\text{m}$,
which indicates a second clay layer.
This agrees well with the historic borehole data described by \textcite{aignerStochasticInversionTransient2025}, 
which is located $1.2\,\text{km}$ south of the Martenhofer Lacke.
The fourth layer not found by \textcite{aignerStochasticInversionTransient2025} could be a second saturated sandy layer
due to its resistivity of $39-40\,\Omega\text{m}$ (\cref{tab:resistivity}), indicating a second aquifer with a lower salinity 
(due to its higher resistivity, than the first sandy layer).

\begin{table}[!ht]
    \centering
    \caption[Subsurface Resistivity Model]{
        The model of the subsurface resistivity obtained through the inversion of the soundings M002 of the first 
        and M050 of the second survey
        }
    \resizebox{\columnwidth}{!}{
        \begin{tabular}{ccc}
            \toprule
            \textbf{Layer number} & \textbf{Thickness ($\text{m}$)} & \textbf{resistivity ($\Omega\text{m}$)}\\
            \midrule
            1 & $4.0\,\text{m}$ & $14-18\,\Omega\text{m}$ \\
            2 & $6.5\,\text{m}$ & $25-31\,\Omega\text{m}$ \\
            3 & $6.0\,\text{m}$ & $7-10\,\Omega\text{m}$ \\
            4 & $\infty$ & $39-42\,\Omega\text{m}$ \\
            \bottomrule
        \end{tabular}
    }
    \label{tab:inversion-model}
\end{table}

The results for ``M052'' (second survey) do not show the resistive second layer described in \cref{tab:inversion-model}.
This could be due to noise introduced to the observed TEM data or variation in the geology along the lake, 
as the soundings ``M002'' (first survey) and ``M052'' (second survey) were collected further north in the Martenhofer Lacke
(\cref{fig:may-map,fig:oct-map}).
The model obtained for sounding ``M028'' (first survey) shows a different resistivity distribution than the other three soundings
discussed in this section as well as the model described by \textcite{aignerStochasticInversionTransient2025}.

We also computed the inversions for all the soundings first with the visually determined $\lambda_\text{opt}$,
then with the mode of all the lambda values ($\lambda_\text{mode}$) seperately for each survey.
For the first survey, where the L-Curve technique worked for $83.33\,\%$ of the soundings, 
the resulting subsurface model of the resistivity obtained with $\lambda_\text{mode}$ showed no significant
differences with the one obtained with $\lambda_\text{opt}$.
This allows for $\lambda_\text{mode}$ to be used for the soundings, where no $\lambda_\text{opt}$ was found using 
the L-Curve method.
For the second survey this technique did not work, which is possibly due to the mode being computed for only
$66.67\,\%$ of the soundings.
The second survey highlights, that distortions in the observed data curves, possibly due to noise, 
tend to lead to a not usable L-Curve, which could indicate not sufficient data qualtity or filtering.
\cref{fig:good-signal-oct}(d) shows that there is great variety in the apparent resistivity curves 
of the soundings of the second survey, even after filtering, and thus filtering should be done for every sounding seperately, 
which makes this measuring condfiguration not feasible for mapping applications.

All inversion plots can be found at \url{https://github.com/pb-tuwien/BSc_Soda_Lakes_Balogh.git}
under "data/2024xxxx/TEM\-data/06\-inversion\_plot/optimised\_M0yy.png" and 
"data/2024xxxx/TEM\-data/06\-inversion\_plot/same\_lambda\_M0yy.png", where xxxx is either 
"0522" for the first or "1008" for the second survey and yy being the sounding number 
(either 01 to 45 or 01 to 66 -- exluding the numbers for the noisy soundings).