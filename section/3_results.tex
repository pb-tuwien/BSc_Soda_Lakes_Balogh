The raw data were plotted in terms of the measured secondary magnetic field ($\frac{\text{V}}{\text{m}^2}$), referred to as ``impulse response'' in this thesis, 
and the conversion to apparent resistivity $\rho_{\text{a}}$ ($\Omega\text{m}$), using \cref{eq:apparent_resistivity}.
The latter reveal larger contrasts and permit an easier visual identification of outliers.
The data gathered in the two surveys was processed separately.

\subsection{Filtering of the Data}
\label{subsec:filtering}
After plotting, the soundings of the first survey (22nd May 2024) were split into two groups:
``Clean'' and ``noisy'' data, with noisy referring to sounding close to the road 
(``M014'', ``M043'', and ``M044'' as seen in \cref{fig:may-map}).
The noisy soundings are likely related to measurements affected by interferences of possible
anthropogenic structures and parasitic electromagnetic fields contminating the data~\parencite{christiansenTransientElectromagneticMethod2006}.
\cref{fig:good-signal-may} shows the clean soundings of the first survey (22nd May 2024) with the $12.5\times12.5\,\text{m}$ loop 
and \cref{fig:err-signal-may} shows the noisy data.
In \cref{fig:good-signal-may}(b) it can be clearly seen that the effects of the turn-off ramp 
as decribed by \textcite{aignerFlexibleSingleLoop2021} influence the measurents until about $8\,\mu\text{s}$, 
which is consistent with the $4.2-8.6\,\mu\text{s}$, found by \textcite{aignerFlexibleSingleLoop2021} for conductive media.
At about $240\,\mu\text{s}$ \cref{fig:good-signal-may}(b) shows how the various curves start having distortions
and \cref{fig:good-signal-may}(a) shows that measured impulse responses are becoming a similar order of magnitude as the measured errors.
For this reason the measurements were filtered by excluding all time gates outside of the time range of $8-210\,\mu\text{s}$ 
as shown in \cref{fig:good-signal-may}(c) and (d).
\cref{fig:good-signal-may}(d) shows the pattern, that the $\rho_{\text{a}}$ values first decrease for most soundings 
and start increasing for all soundings for a time larger than $30\,\mu\text{s}$.

\begin{figure*}[!ht]
    \centering
    \includegraphics[width=0.8\linewidth]{data/20240522/TEM-data/04-first_look/20240522_good_soundings.png}
    \caption[Response Curves of the May Survey]{
        Raw and filtered impulse responses and apparent resistivities from the 22nd May 2024 survey, where anomalies are excluded.
        The first sounding (M001) is colored dark purple, the last sounding (M045) is yellow, 
        and all soundings in between are assigned a color from a uniform distribution between the two border colors.
        Subfigure (a) shows the raw impuls response as well as the measured errors as dashed lines, 
        (b) shows the computed apparent resistivities, 
        (c) show the filtered impulse response, the measured error as dashed lines, and greyed out raw data as comparison,
        (d) shows the resulting filtered apparent resistivities and greyed out raw values.
    }
    \label{fig:good-signal-may}
\end{figure*}

The apparent resistivities of the noisy soundings as seen in \cref{fig:err-signal-may}(b) 
show no clear pattern due to high distortions in the curves.
The anomalous measurements shown in \cref{fig:err-signal-may} were still processed, to test how well they compare to the other soundings.
In order to make it more comparable the same time range of $8-210\,\mu\text{s}$ was used for the filtering (\cref{fig:err-signal-may}(d)).

\begin{figure*}[!ht]
    \centering
    \includegraphics[width=0.8\linewidth]{data/20240522/TEM-data/04-first_look/20240522_err_soundings.png}
    \caption[Anomalous Responses of the May Survey]{
        The anomalous raw and filtered impulse responses and apparent resistivities from the 22nd May 2024 survey.
        Subfigure (a) shows the raw impuls response as well as the measured errors as dashed lines, 
        (b) shows the computed apparent resistivities, 
        (c) show the filtered impulse response, the measured error as dashed lines, and greyed out raw data as comparison,
        (d) shows the resulting filtered apparent resistivities and greyed out raw values.
    }
    \label{fig:err-signal-may}
\end{figure*}

The clean soundings (\cref{fig:good-signal-oct}) and anomalies (\cref{fig:err-signal-oct}) of the second survey (8th October 2024) 
with the $6.25\times6.25\,\text{m}$ loop were processed in a similar way.
Comparing \cref{fig:good-signal-oct}(b) with \cref{fig:good-signal-may}(b) shows that the effects of the turn-off ramp both end at $8\,\mu\text{s}$, 
while the distortions in the data from the second survey start earlier at $100-110\,\mu\text{s}$.
The impulse response also becomes the same order of magnitude as the measured error~(\cref{fig:good-signal-oct}(a)) at the same time~($110\,\mu\text{s}$).
Thus all the time gates outside the time range $8-110\,\mu\text{s}$ were filteredas, shown in \cref{fig:good-signal-oct}(c) and (d).
\cref{fig:good-signal-oct}(d) shows that in contrast to the soundings with the larger loop (\cref{fig:good-signal-may}(d)), 
the $\rho_\text{a}$ values do not decrease in the early times, but also start increasing after $30\,\mu\text{s}$.

\begin{figure*}[!ht]
    \centering
    \includegraphics[width=0.8\linewidth]{data/20241008/TEM-data/04-first_look/20241008_good_soundings.png}
    \caption[Response Curves of the October Survey]{
        Raw and filtered impulse responses and apparent resistivities from the 8th October 2024 survey, where anomalies are excluded.
        The first sounding (M001) is colored dark purple, the last sounding (M066) is yellow, 
        and all soundings in between are assigned a color from a uniform distribution between the two border colors.
        Subfigure (a) shows the raw impuls response as well as the measured errors as dashed lines, 
        (b) shows the computed apparent resistivities, 
        (c) show the filtered impulse response, the measured error as dashed lines, and greyed out raw data as comparison,
        (d) shows the resulting filtered apparent resistivities and greyed out raw values.
        }
        \label{fig:good-signal-oct}
    \end{figure*}
    
The noisy soundings shown in \cref{fig:err-signal-oct} can be divided into two groups:
Measurements close to man-made structures, such as the road (``M002'', ``M009'', ``M010'', and ``M024'' as seen in \cref{fig:may-map}), 
which show similarities to the noisy soundings of the first survey (\cref{fig:err-signal-may}(b)).
``M002'', ``M009'', and ``M010'' show significantly lower apparent resistivity values, than the other soundings --
at $40\,\mu\text{s}$ $10\,\Omega\text{m}$~(\cref{fig:err-signal-oct}) instead of $20\,\Omega\text{m}$~(\cref{fig:good-signal-oct}).
This could be due to coupling effects, as the impulse response curves (\cref{fig:err-signal-oct}(a)) look similar to the 
example for capacitive coupling described by \textcite{christiansenTransientElectromagneticMethod2006}.
``M024'' shows higher a apparent resistivitiy of $30\,\Omega\text{m}$ at $40\,\mu\text{s}$ and a much steeper increase in
the $\rho_\text{a}$ values~(\cref{fig:err-signal-oct})

Measurements within the lake (``M057''-``M065'', except for ``M062'', as seen in \cref{fig:may-map}), 
which show effects of the turn-off ramp until $12\,\mu\text{s}$ (\cref{fig:err-signal-oct}(b)).
For these sounding the loop was submerged in the lake (up to $30\,\text{cm}$ deep), 
where the water is a good conductor due to its high salinity with resistivities of $0.78-7.16\,\Omega\text{m}$~\parencite{borosHowCanUnique2025}.
This also breaks the assumption of a layered half-space, 
which is necessary for the inversion of TEM data~\parencite{christiansenTransientElectromagneticMethod2006}, and 
thus all soundings within the lake (marked orange in \cref{fig:oct-map}) were excluded from further processing.

\begin{figure*}[!ht]
    \centering
    \includegraphics[width=0.8\linewidth]{data/20241008/TEM-data/04-first_look/20241008_err_soundings.png}
    \caption[Anomalous Responses of the October Survey]{
        The anomalous raw and filtered impulse responses and apparent resistivities from the 8th October 2024 survey.
        Subfigure (a) shows the raw impuls response as well as the measured errors as dashed lines, 
        (b) shows the computed apparent resistivities, 
        (c) show the filtered impulse response, the measured error as dashed lines, and greyed out raw data as comparison,
        (d) shows the resulting filtered apparent resistivities and greyed out raw values.
    }
    \label{fig:err-signal-oct}
\end{figure*}

\subsection{Inversion Parameters}
\label{subsec:inversion-parameters}

In this section we evaluate the effect in the inversion due to varying parameters, 
which are necessary for the algorithm by \textcite{aignerFlexibleSingleLoop2021}:
The relative error vector, range of $\lambda$ values, and the layer distribution.
The relative error vector specifies the corresponding relative error for each recorded impulse response of a time gate.
It can be derived from the absolute errors, which the device records automatically or set manually.
\textcite{aignerSensitivityAnalysisInverted2024} conducted a survey at the same location with a $50.0\times50.0\,\text{m}$ loop
and found that the device overestimates the error.
Thus \textcite{aignerSensitivityAnalysisInverted2024} used a constant $1.5\,\%$ relative error vector for the inversion.
For the data collected in both surveys of this thesis, the inversions were run defining an relative error vector of $1.5\,\%$.

The inversion algorithm of \textcite{aignerFlexibleSingleLoop2021} also requires a regularisation parameter $\lambda$.
This thesis focusses on finding an optimal $\lambda$ by using the L-Curve method.
The L-Curve method was implemented in a way, that it requires an interval of two $\lambda$ values,
which are used to create a range $20$ of logarithmically spaced $\lambda$ values.
These are then used to compute $20$ inversions of which the corresponding RMS misfit 
and model roughness is plotted to create the L-Curve.
The study from \textcite{aignerSensitivityAnalysisInverted2024} uses $\lambda$ values between $40.5-50$ for the inversion.
To allow divergence from these values, a lower bound of $5$ and an upper bound of $100$ was chosen for this thesis.

The layer distribution specifies the number and thicknesses of the layers characterising
the subsurface model, which is used for the inversion.
\textcite{welkensComparisonSubsurfaceGeophysical2025} shows that a layer model with $1\,\text{m}$-Layers until $5\,\text{m}$ depth 
and $1.5\,\text{m}$-Layers below that, until a maximal depth of $10\,\text{m}$ optimises the inversion
speed and model fit for data gathered with a $6.25\times6.25\,\text{m}$ loop and $4.1\,\text{A}$ of current.
For a first survey of this thesis a $12.5\times12.5\,\text{m}$ loop was 
and using \cref{fig:inv-depth-10m}(c) shows that the inversion algorithm struggles to fit the late time data, 
when using the layer distribution suggested by \textcite{welkensComparisonSubsurfaceGeophysical2025}.
\cref{fig:inv-depth-20m}(c) shows that this issue can be resolved by choosing a maximal depth of $20\,\text{m}$ 
and keeping the other layer thickness suggestions from \textcite{welkensComparisonSubsurfaceGeophysical2025}.
The visual differences between \cref{fig:inv-depth-10m}(c) and \cref{fig:inv-depth-20m}(c) are minimal, 
and thus the relative root-mean-square (RMS) of the misfit between modelled and observed data was used to quantify the imporvement,
which decreased from $2.32\,\%$ to $2.09\,\%$ using $20\,\text{m}$ instead of $10\,\text{m}$ as the maximal depth.

\begin{figure*}[ht]
    \centering
    \includegraphics[width=0.8\linewidth]{data/parameter_testing/M028_max_depth_10m.png}
    \caption[Inversion With 10m Maximum Depth]{
        Inversion results for a representative sounding (M028) for the first survey (22nd May 2024)
        with the layer distribution of $1\,\text{m}$ thicknesses until $5\,\text{m}$ depth and
        $1.5\,\text{m}$ until $10\,\text{m}$ depth.
        Subfigure (a) shows the final subsurface model of the resistivity, 
        (b) shows the comparison between the modelled and measured impulse response as well as the measured error as a dashed line,
        (c) shows the comparison between the modelled and measured apparent resistivities.
    }
    \label{fig:inv-depth-10m}
\end{figure*}

\begin{figure*}[ht]
    \centering
    \includegraphics[width=0.8\linewidth]{data/parameter_testing/M028_max_depth_20m.png}
    \caption[Inversion With 20m Maximum Depth]{
        Inversion results for a representative sounding (M028) for the first survey (22nd May 2024)
        with the layer distribution of $1\,\text{m}$ thicknesses until $5\,\text{m}$ depth and
        $1.5\,\text{m}$ until $20\,\text{m}$ depth.
        Subfigure (a) shows the final subsurface model of the resistivity, 
        (b) shows the comparison between the modelled and measured impulse response as well as the measured error as a dashed line,
        (c) shows the comparison between the modelled and measured apparent resistivities.
    }
    \label{fig:inv-depth-20m}
\end{figure*}

\FloatBarrier
\subsection{The L-Curve Method}
\label{subsec:manual-l-curve}

The L-Curve technique can be used to plot the model roughness against the model misfit for varying regularisation parameter ($\lambda$) values, 
which can be used to find the optimal lambda $\lambda_\text{opt}$~\parencite{hansenAnalysisDiscreteIllPosed1992}.
This can be done by visually finding the ``corner'' of the L-curve, 
which were computed with the parameters from \cref{subsec:inversion-parameters}:
$1.5\,\%$ as the relative error, $20$ logarithmically spaced $\lambda$ values in the range $5-100$, 
and the layer distribution from \textcite{welkensComparisonSubsurfaceGeophysical2025} but with a maximal depth of $20\,\text{m}$.
For the first survey the L-Curves were computed for the clean as well as noisy soundings, 
where we found no significant visual difference between the clean and noisy data, and thus
excluded all noisy data from further analysis.
The remaining $43$ soundings of the first survey were manually analysed 
and we were able to find an optimal lambda $\lambda_\text{opt}$ for $83.3\,\%$ of the soundings.
Doing the same for the $54$ clean soundings of the second survey resulted in finding a $\lambda_\text{opt}$ for $66.7\,\%$
of the soundings.

\cref{fig:lcurve-ideal,fig:lcurve-usable} show two L-Curves, which allowed finding an $\lambda_\text{opt}$.
\cref{fig:lcurve-ideal} shows the L-Curve for the sounding ``M028'' of the first survey 
and illustrates the expected L-Curve as it is described by \textcite{hansenAnalysisDiscreteIllPosed1992}.
Based on \cref{fig:lcurve-ideal} we determined the optimal lambda for this sounding to be at $13$,
which minimises both the RMS misfit and the model roughness.
\cref{fig:lcurve-usable} shows the L-Curve for the sounding ``M003'' of the second survey 
and was selected to showcase how the computed L-Curve can deviate from the expected ideal, 
but still provide usefull information.
\cref{fig:lcurve-usable} shows contrary to what is expected, 
the RMS misfit increases between lambdas $8-5$ and decreases between $63-100$, 
but the curve still has the characteristic ``corner'' at the $\lambda$ value $8$,
which we determined to be the optimal lambda.

\begin{figure*}[ht]
    \centering
    \includegraphics[width=0.55\linewidth]{data/20240522/TEM-data/07-inversion_analysis/l_curve_M028.png}
    \caption[Example of an ideal L-Curve]{
        L-Curve of the sounding M028 in the first survey (22nd May 2024), 
        which shows the expected characteristics of a L-Curve.
    }
    \label{fig:lcurve-ideal}
\end{figure*}

\begin{figure*}[ht]
    \centering
    \includegraphics[width=0.55\linewidth]{data/20241008/TEM-data/07-inversion_analysis/l_curve_M003.png}
    \caption[Example of an usable L-Curve]{
        L-Curve of the sounding M003 in the second survey (8th October 2024), 
        which shows a L-Curve that diverges from the expected, 
        but still can be used to find an optimal lambda.
    }
    \label{fig:lcurve-usable}
\end{figure*}

\cref{fig:lcurve-inverted,fig:lcurve-bad} show two examples of the L-Curve method not working to find an
optimal lambda for TEM soundings.
Generally for an increasing $\lambda$ a descreasing model roughness 
and an increasing model misfit is expected~\parencite{hansenAnalysisDiscreteIllPosed1992}, 
but \cref{fig:lcurve-inverted}, which was caclulated for the sounding ``M045'' of the first survey, 
shows a decreasing model misfit (RMS misfit) for an increasing $\lambda$.
It is plausible that this behaviour changes with further increases in the $\lambda$ values, 
as it happens in \cref{fig:lcurve-usable} between $8-53$.
\cref{fig:lcurve-bad} shows an example, which was computed for the sounding ``M019'' of the second survey, 
where the L-Curve method produced no cohesive curve, but rather collection of seemingly randomly positioned points.

\begin{figure*}[ht]
    \centering
    \includegraphics[width=0.55\linewidth]{data/20240522/TEM-data/07-inversion_analysis/l_curve_M045.png}
    \caption[Example of an inverted L-Curve]{
        L-Curve of the sounding M045 in the first survey (22nd May 2024), 
        which shows an increasing RMS with an increasing roughness, 
        which is the opposite to the expected characteristics of a L-Curve.
    }
    \label{fig:lcurve-inverted}
\end{figure*}

\begin{figure*}[ht]
    \centering
    \includegraphics[width=0.55\linewidth]{data/20241008/TEM-data/07-inversion_analysis/l_curve_M019.png}
    \caption[Example of a bad L-Curve]{
        ``L-Curve'' of the sounding M019 in the second survey (8th October 2024), 
        which shows a seemingly random collection of points instead of a L-Curve.
    }
    \label{fig:lcurve-bad}
\end{figure*}

\subsection{Finding an Optimal Lambda}
\label{subsec:optimal-lambda}

The clean soundings of both surveys were used to test the three search algorithms presented in \cref{subsubsec:inversion}:
All algorithms try finding the $\lambda$ corresponding to the point on the L-CUrve with the maximal curvature.
The ``cubic-spline'' based approach fits a cubic spline function, 
which is then used to compute the curvatures of each point~\parencite{lloydUseLcurveMethod1997}.
The ``gradient-based'' method utilises the \texttt{gradient()} function from the \texttt{numpy} library to achieve the same.
The ``golden section search'' is an iterative search algorithm based on \textcite{cultreraSimpleAlgorithmFind2020}.

In early versions all three search algorithms performed poorly 
and we attributed this to the fact that the RMS misfit and the model roughness are $5-6$ orders of magnitude apart,
which can be seen when looking at the axis of \cref{fig:lcurve-usable}.
\cref{fig:example-norm-false} shows how all three search algorithms return a $\lambda$ between $41-62$,
which is higher than the visually determined $\lambda_\text{opt}$ at $13$.
To counteract this a linear normalisation, using the formula
$\mathbf{a}_i = \frac{\mathbf{a}_\text{i}-\mathbf{a}_\text{min}}{\mathbf{a}_\text{max}-\mathbf{a}_\text{min}}$, was 
applied to both RMS and roughness.
\cref{fig:example-norm-true} shows how with the normalisation all three search algorithms return a value ($12-15$) around 
the visual optimum of $13$. Hence all further searches for an optimal lambda were carried out with the normalisation.

\begin{figure*}[ht]
    \centering
    \includegraphics[width=0.55\linewidth]{data/parameter_testing/M028_normalization_false.png}
    \caption[Lambda Search Without Normalisation]{
        Automatically searching for an optimal lambda of the sounding M028 in the first survey (22nd May 2024) 
        without applying a normalisation to the L-Curve.
    }
    \label{fig:example-norm-false}
\end{figure*}

\begin{figure*}[ht]
    \centering
    \includegraphics[width=0.55\linewidth]{data/parameter_testing/M028_normalization_true.png}
    \caption[Lambda Search With Normalisation]{
        Automatically searching for an optimal lambda of the sounding M028 in the first survey (22nd May 2024) 
        while applying a normalisation to the L-Curve.
    }
    \label{fig:example-norm-true}
\end{figure*}

\cref{fig:example-norm-true} demontrates that all three search algorithms are able to find the optimal lambda for a 
L-Curve, which resembles an ideal L-Curve as described by \textcite{hansenAnalysisDiscreteIllPosed1992}.
\cref{fig:search-working} shows that all three search algorithms find the optimal lambda at $8$ when there is a clear
point with a maximal curvature, even if the L-Curve deviates from the expected ideal.
\cref{fig:search-not-working} demonstrated how the various search algorithm find different ``optimal'' lambda values, 
when the curvature of the computed L-Curve has no distinct maximum, 
which is exemplified by sounding ``M020'' of the first survey.
When inspecting the L-Curve shown in \cref{fig:search-not-working}, we determined $\lambda_\text{opt}$ to be $15$:
The ``cubic-spline'' and ``gradient'' based search algorithms overestimated $\lambda_\text{opt}$ to be between $45-53$,
while the ``golden section search'' underestimated $\lambda_\text{opt}$ to be $7$.

\begin{figure*}[ht]
    \centering
    \includegraphics[width=0.55\linewidth]{data/20241008/TEM-data/07-inversion_analysis/comparison_M003.png}
    \caption[Example Lambda Search Not Working]{
        Automatically searching for an optimal lambda of the sounding M003 in the second survey (8th October 2024), 
        which showcases how the different search algorithms perform well, even when the computed L-Curve deviates from the theoretical ideal,
        but some expected characteristic of an L-curve can be found.
    }
    \label{fig:search-working}
\end{figure*}

\begin{figure*}[ht]
    \centering
    \includegraphics[width=0.55\linewidth]{data/20240522/TEM-data/07-inversion_analysis/comparison_M020.png}
    \caption[Example Lambda Search Working With Non-Ideal L-Curve]{
        Automatically searching for an optimal lambda of the sounding M020 in the first survey (22nd May 2024), 
        which showcases how performance the different search algorithms deviate.
    }
    \label{fig:search-not-working}
\end{figure*}

These examples highlight the strenghts and shortcomings of each search algorithm. 
\cref{tab:search-summary} shows a summary of how the algorithms performed overall 
and comparing this to a visual inspection of the generated L-Curve.
The L-Curve method worked better at \colorbox{yellow}{revised until here}

In both surveys 7 out of 111 soundings produce a curve, which can not
be used to find an optimal lambda, even when visually inspected.
The golden section search found the least optimal lambdas, only 57 out of 111, but when it worked, it returned the most
accurate results by finding the optimal lambda accurately down to first decimal point.
The cubic spline algorithm was more reliable overall by finding 83 out of 111 optimal lambdas, but as
shows it also struggles with suboptimal L-Curves.
In our analysis the gradient based curvature analysis yielded the most promising results
by finding 89 out of 111 optimal lambda values, 
hence it was used to find the optimal lambda for each sounding respectively.
As all search algorithms maximise the curvature of the L-Curve to find the lambda, all 
of them struggle with curves with minimal curvature or even linear graphs.
\cref{tab:search-summary} also shows that there are minimal differences in the produced L-Curves 
as well as the effectiveness of the automated search algorithms between the two surveys 
and thus the two measuring configurations.
All comparison plots can be found at \url{https://github.com/pb-tuwien/BSc_Soda_Lakes_Balogh.git}
under "data/2024xxxx/TEM\-data/07\-inversion\_analysis/comparison\_M0yy.png", where xxxx is either 
"0522" for the first or "1008" for the second survey and yy being the sounding number (either 01 to 45 or 01 to 66).

\begin{table*}[!ht]
    \centering
    \caption[Summary of the search for an optimal Lambda]{
        Summary of the search for an optimal lambda for each sounding, comparing the visual identification 
        with the various automated search algorithms.
        }
    \begin{tabular}{ccc}
        \toprule
        \textbf{Search Type} & \textbf{Sucessrate (\%)} & \textbf{Mean relative deviation (\%)}\\
        \midrule
        \textbf{22nd May 2024} \\
        Visual & 83.33 & - \\
        Cubic Spline & 60.00 & 32.66 \\
        Gradient Based & 71.43 & 15.89 \\
        Golden Section & 74.29 & 17.55 \\
        \textbf{8th October 2024} \\
        Visual & 66.67 & - \\
        Cubic Spline & 33.33 & 65.21 \\
        Gradient Based & 47.22 & 39.57 \\
        Golden Section & 41.67 & 118.28 \\
        \bottomrule
    \end{tabular}
    % \end{tblr}
    \label{tab:search-summary}
\end{table*}

% \FloatBarrier
% \subsection{Comparing Inversion Results}\label{subsec:inversion-results}
% With these findings, we finally obtained the necessary parameters for running the inversions, 
% which enables a comparison of different resistivity models obtained from the various measuring configurations,
% as described in \cref{tab:sounding-parameters}. 
% Because the late time data was filtered due to a high noise level, the different time keys are not
% considered and only the different injected currents and the loop sizes are compared.
% Ideally the compared soundings, would be at the same location, which is why we chose the soundings
% M028 (shown in \cref{fig:inv-big-small}) of the first and M052 (shown in \cref{fig:inv-small-big}) of the second survey, 
% which use the $12.5\times12.5\,\text{m}$ loop with $1.0\,\text{A}$ of current 
% and $6.25\times6.25\,\text{m}$ loop with $4.1\,\text{A}$ of current.
% \cref{fig:may-map,fig:oct-map} show that these two broadly align.
% For the configuration $12.5\times12.5\,\text{m}$ loop with $4.1\,\text{A}$ of current
% we do not have a sounding at the same location, but sounding M002 (shown in \cref{fig:inv-big-big}) of the first survey
% is also in the north of the Martenhofer Lacke (\cref{fig:may-map}).
% \cref{fig:inv-big-small}(c) shows how with the large loop three distinct layers can be found: A $5\,\text{m}$-layer 
% with $17.5\,\Omega\text{m}$, a second $11\,\text{m}$-layer with $13.5\,\Omega\text{m}$, 
% and the start of a third layer with $22\,\Omega\text{m}$.
% The model seems to fit the data adequately as can be seen in \cref{fig:inv-big-small}(a) and (b), 
% which is also supported by a relative RMS misfit of $4.77\,\%$.

% \begin{figure*}[ht]
%     \centering
%     \includegraphics[width=\linewidth]{data/20240522/TEM-data/06-inversion_plot/location_2_12.5_M028.png}
%     \caption[Inversion of Sounding with the Large Loop and Small Current]{
%         Optimised inversion of the sounding M028 of the first survey, which was measured with a $12.5\times12.5\,\text{m}$ loop
%         and $1.0\,\text{A}$ of current.
%         This sounding was done at the same location as M052 of the second survey.
%         Subfigure (a) shows the comparison between modelled and measured impulse response as well as the measured error,
%         (b) shows the comparison between modelled and measured apparent resistivities,
%         (c) shows the final subsurface model of the resistivity,
%         (d) shows the L-curve of this sounding with the chosen lambda value marked in orange.
%         }
%         \label{fig:inv-big-small}
%     \end{figure*}
    
% For the same location five moths later the small loop with $4.1\,\text{A}$ produces a model with only
% two discernable layers, as seen in \cref{fig:inv-small-big}(c): A $5\,\text{m}$-layer with $16\,\Omega\text{m}$ and
% the start of a second layer with $24\,\Omega\text{m}$.
% The second layer is not distinctly visible, but rather a gradual increase, 
% which corresponds with the almost linearly increasing apparent resistivity curve in \cref{fig:inv-small-big}(b).
% \cref{fig:inv-small-big}(a) and (b) also show that the last measured data point possibly indicates s change in the slope
% and thus a new layer~\parencite{fittermanTransientElectromagneticSounding1986}, but it could also be data noise.

% \begin{figure*}[ht]
%     \centering
%     \includegraphics[width=\linewidth]{data/20241008/TEM-data/06-inversion_plot/location_2_6.25_M052.png}
%     \caption[Inversion of Sounding with the Small Loop and Large Current]{
%         Optimised inversion of the sounding M052 of the second survey, which was measured with a $6.25\times6.25\,\text{m}$ loop
%         and $4.1\,\text{A}$ of current.
%         This sounding was done at the same location as M028 of the first survey.
%         Subfigure (a) shows the comparison between modelled and measured impulse response as well as the measured error,
%         (b) shows the comparison between modelled and measured apparent resistivities,
%         (c) shows the final subsurface model of the resistivity,
%         (d) shows the L-curve of this sounding with the chosen lambda value marked in orange.
%     }
%     \label{fig:inv-small-big}
% \end{figure*}

% \cref{fig:inv-big-big} represents the measurements made with the large loop as well as $4.1\,\text{A}$ 
% and in (c) similar three layers can be seen as in \cref{fig:inv-big-small}(c) but with differing resistivity values:
% A $5\,\text{m}$-layer with $31\,\Omega\text{m}$,
% a second $12\,\text{m}$-layer with $13.5\,\Omega\text{m}$, and
% the start of a third layer with $30\,\Omega\text{m}$.
% \cref{fig:inv-big-big}(b) shows that the model does not quite fit the measured data, 
% but follows the general trend of the curve.

% \begin{figure*}[ht]
%     \centering
%     \includegraphics[width=\linewidth]{data/20240522/TEM-data/06-inversion_plot/optimised_M002.png}
%     \caption[Inversion of Sounding with the Large Loop and Large Current]{
%         Optimised inversion of the sounding M002 of the first survey, which was measured with a $12.5\times12.5\,\text{m}$ loop
%         and $4.1\,\text{A}$ of current.
%         Subfigure (a) shows the comparison between modelled and measured impulse response as well as the measured error,
%         (b) shows the comparison between modelled and measured apparent resistivities,
%         (c) shows the final subsurface model of the resistivity,
%         (d) shows the L-curve of this sounding with the chosen lambda value marked in orange.
%     }
%     \label{fig:inv-big-big}
% \end{figure*}

% We found that the measurements with the $12.5\times12.5\,\text{m}$ loop were able
% to discern the same three layers regarding the thicknesses, which \textcite{aignerSensitivityAnalysisInverted2024} found
% in a borehole at the soda lakes, which is located about $1.5\,\text{km}$ north-east of the Martenhofer Lacke.
% \textcite{aignerSensitivityAnalysisInverted2024} describes a $1.6\,\text{m}$ clay-silt layer, 
% followed by a $5.4\,\text{m}$ sandy gravel aquifer, 
% and a second clay-silt layer until the maximal drilled depth of $10\,\text{m}$.
% \cref{fig:inv-big-small,fig:inv-big-big} both show in subfigure (c) that the first two layers of the
% borehole can not be differentiated and are show as one layer with the resistivities of $31\,\Omega\text{m}$
% and $17.5\,\Omega\text{m}$ respectively. 
% These diffences could be caused by differing clay-silt layer and aquifer thicknesses,
% which also changes the influence of each over the resistivity value representing the compined resistivities of both layers.
% The second layer present in the subfigures (c) of \cref{fig:inv-big-small,fig:inv-big-big}
% shows a layer with a resistivity of about $13.5\,\Omega\text{m}$, which points to a clay-silt layer when compared with
% the values in \cref{tab:resistivity} (\colorbox{yellow}{Referencing table with common resistivity}
% \colorbox{yellow}{from the State of the Art}).
% This agrees well with the borehole data described by \textcite{aignerSensitivityAnalysisInverted2024}.
% The third layer with the resistivity of $22\,\Omega\text{m}$ and $30\,\Omega\text{m}$ respectively, 
% could indicate a second aquifer, which was also detected by \textcite{aignerSensitivityAnalysisInverted2024}, who found
% that a second aquifer with the resistivity value of $30\,\Omega\text{m}$ at a depth below $35\,\text{m}$.

% The model created through the data gathered with the $6.25\times6.25\,\text{m}$ loop, as seen in \cref{fig:inv-small-big}, only shows the first 
% two layers and is unable to resolve the third layer. 
% The advantage of being able to resolve shallow layers with a smaller loop~\parencite{kirsch2006groundwater},
% is counteracted by the effects of turn-off ramp as described by \textcite{aignerFlexibleSingleLoop2021} and thus the need to discard the early time data.
% The results produced with the larger loop and the current of $4.1\,\text{A}$ seem to reproduce the results of earlier studies like
% \textcite{aignerSensitivityAnalysisInverted2024} the best out of the configurations tested in this work.