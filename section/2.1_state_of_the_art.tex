Even with these limitations the TEM method has found widespread application 
in environmental applications like groundwater exploration~\parencite{danielsen_application_2003,sorensen_skytemnew_2004,auken_ttem_2019},
delineating saline intrusions into the groundwater~\parencite{gomez_alluvial_2019,gonzales_amaya_delimiting_2018}, 
characterising the subsurface below continental water bodies~\parencite{aigner_flexible_2021}, 
and the detection of karstic features~\parencite{zhou_multi-turn_2022, su_response_2024}.

\subsubsection{Environmental Applications of TEM}
\label{subsubsec:application-of-tem}

\textcite{danielsen_application_2003} used the TEM method for the exploration of buried valleys as potential aquifers in Denmark.
The study introduced two new TEM systems based on the Geonics PROTEM: 
The high moment transient electromagnetic (HiTEM) system for a deeper investigation depth,
and the pulled array Transient electromagnetic (PATEM) system for a higher lateral resolution.
The HiTEM system uses a combination of a $30\times30\,\text{m}$ transmitter loop 
and a current of $75\,\text{m}$ to achieve a depth of investigation (DOI) of up to $300\,\text{m}$~\parencite{danielsen_application_2003}.
To mitigate distortion effects a HiTEM sounding is split into two parts:
A $70\,\text{m}$ offset loop configuration for the later time data, and
 a central loop configuration with a reduced current of $2.5\,\text{A}$ for the early times~\parencite{danielsen_application_2003}.
A mutually constrained inversion (MCI) like in \textcite{auken_mutually_2005} is used to invert the 
combination of the data gathered with the two different configurations~\parencite{danielsen_application_2003}.
\textcite{danielsen_application_2003} developed the PATEM system to use a $3\times5\,\text{m}$ transmitter loop 
on a wheeled frame with an offset receiver loop, 
which allows for continuous measurements along a profile and a DOI of $100-150\,\text{m}$.
To achieve a high DOI and useful near surface information, 
the PATEM system allows for a transmitter configuration with either 2 turns with $16\,\text{A}$ 
or 8 turns with $40-50\,\text{A}$~\parencite{danielsen_application_2003}.
That study also touches on the problems resulting from coupling with man-made structures.
Coupling noise can be described as the noise introduced through induced currents in man-made structures, 
which changes the decay characteristics of the secondary magnetic field,
and can be divided into galvanic and capacitive~\parencite{christiansen_transient_2006}.
Galvanic coupling is caused by grounded conductors like power lines 
and causes an underestimation of the resistivity~\parencite{christiansen_transient_2006}.
Capacitive coupling is caused by current being generated in the conductor 
and leaking into the ground through an insulation, which leads to an oscillating signal~\parencite{christiansen_transient_2006}.
For this reason the \textcite{danielsen_application_2003} recommend keeping a distance of about $150\,\text{m}$ from 
underground cables or pipes when the earth has a resistivity of $40-50\,\Omega\text{m}$.

Similarly to the PATEM system, \textcite{auken_ttem_2019} propose a towed transient electromagnetic (tTEM) 
system for an efficient 3D mapping of the subsurface.
The tTEM system utilises a $2\times4\,\text{m}$ transmitter loop mounted on a non-conductive sled towed by an vehicle, 
which enables a production rate of about $1\,\text{km}^2$ per day~\parencite{auken_ttem_2019}.
The receiver loop is towed at $9\,\text{m}$ offset from the transmitter~\parencite{auken_ttem_2019}.
Like the PATEM system~\parencite{danielsen_application_2003}, 
the tTEM system~\parencite{auken_ttem_2019} permits the use of two different currents ($2.8~\text{and}~30\,\text{A}$) 
in order to gain a relatively high DOI of up to $70\,\text{m}$,
while still allowing the investigation of shallow depths.
\textcite{auken_ttem_2019} also highlight the importance of considering the coupling effects of the system with conductive objects
and it tests the system in an environment with a high resistivity of the subsurface ($>600\,\Omega\text{m}$).
Under these circumstances the signal caused by coupling can be observed in isolation
and the study found that a minimal distance of $3\,\text{m}$ between transmitter and 
vehicle was necessary to mitigate coupling effects~\parencite{auken_ttem_2019}.

Electromagnetic methods do not require direct contact with the subsurface, 
which allowed for the development of airborne electromagnetic methods (AEM)~\parencite{christiansen_transient_2006}.
\textcite{sorensen_skytemnew_2004} introduced the SkyTEM system as an alternative to ground based TEM systems.
The SkyTEM system uses a helicopter to carry a $12.5\times12.5\,\text{m}$ transmitter loop with 4 turns and 
a receiver loop ($0.5\times0.5\,\text{m}$) in a central configuration~\parencite{sorensen_skytemnew_2004}.
Just like the PATEM and HiTEM systems~\parencite{danielsen_application_2003}, 
a low moment and a high moment configuration are used to achieve a high DOI, 
while still resolving the near surface layers~\parencite{sorensen_skytemnew_2004}.
In the low moment configuration the current of $35\,\text{A}$ only flows through one turn and 
for the high moment $50\,\text{A}$ are used with all 4 turns~\parencite{sorensen_skytemnew_2004}.
\textcite{sorensen_skytemnew_2004} validated the SkyTEM system against a ground based TEM system 
with a transmitter loop size of $40\times40\,\text{m}$, 
showing a good agreement (below 5\% deviation).
This system is able to cover a larger area than traditional ground based systems in the same amount of time, 
while still being able to resolve underground structures,
such as buried valleys~\parencite{sorensen_skytemnew_2004}.

To investigate the subsurface below continental bodies of water, 
\textcite{aigner_flexible_2021} propose a flexible single loop system which can be towed by a boat.
This is made possible through the TEM-FAST Technology, 
which allows for one loop to function as the transmitter and receiver antenna~\parencite{barsukov_chapter_2006}.
The single loop is kept afloat by several PVC pipe segments, that keep it in a circular shape, 
which allows the system to be moved around the lake with ease~\parencite{aigner_flexible_2021}.
Using multiple pipe segments allows for different loop sizes and thus different investigation depths~\parencite{aigner_flexible_2021}.
\textcite{aigner_flexible_2021} used the TEM-FAST~48 system by Applied Electromagnetic Research (AEMR) 
to investigate the subsurface of the Lake Langau in Austria.
A current of $4\,\text{A}$ and loops with radii between $6.2-11.9\,\text{m}$ were used, 
leading to an investigation depth between $6.2-50.0\,\text{m}$,
which was sufficient to detect sedimentary layers below the lake~\parencite{aigner_flexible_2021}.
\textcite{aigner_flexible_2021} highlight the importance of understanding how long the turn-off time of the transmitter 
for a single-loop setup is, to be able to properly interpret early time data.
When the transmitter is turned off, the current in the loop takes a certain amount of time to decay, 
which is called the ``turn-off ramp''~\parencite{christiansen_transient_2006}.
As the transmitter and receiver antennae are the same loop, the turn-off ramp affect the early time readings,
which \textcite{barsukov_chapter_2006} also describes as the ``self-transient process''.
Visually the effects of the turn-off ramp can be seen as an increase of the apparent resistivity values in the early times
independent from the structure of the electrical properties in the subsurface~\parencite{aigner_flexible_2021}.
\textcite{aigner_flexible_2021} measured the time, which turn-off ramp to lasted, 
(minimum effective time) for the TEM-FAST~48 to be between $4.2-10.4\,\mu\text{s}$ -- 
depending on loop size and resistivity of the subsurface.
Using this information, a formula was derived to find the minimum effective sounding depth:

\begin{equation}
    h_{\text{eff}} = \sqrt{t_{\text{eff}}\overline{\rho}}
    \label{eq:h_eff}
\end{equation}

where $h_{\text{eff}}$ is the minimum effective sounding depth, $t_{\text{eff}}$ is the minimum effective time
and $\overline{\rho}$ is the average resistivity of the smooth subsurface model~\parencite{aigner_flexible_2021}.

The study by \textcite{aigner_flexible_2021} also showcases two different approaches to finding the DOI\@.
The first uses different starting models for the inversion assuming that the DOI is reached
when the data does not influence the inverted model anymore 
and will keep the values of the starting model~\parencite{aigner_flexible_2021}.
The second approach is based on the formula:

\begin{equation}
    \text{DOI} \approx 0.55(\frac{M\times\overline{\rho}}{\eta})
    \label{eq:doi}
\end{equation}

where $M$ is the magnetic moment, $\overline{\rho}$ is the average resistivity of the smooth subsurface model 
and $\eta$ is the noise level~\parencite{aigner_flexible_2021}.
Both methods agree on a DOI ranging between 20 and 50~m depending on the loop size~\parencite{aigner_flexible_2021}.

\textcite{gonzales_amaya_delimiting_2018} show the potential of the TEM method 
for the exploration of groundwater in the Punata alluvial fan.
This aquifer is an important water source for the region, 
but it has zones with high salinity, which poses challenges in its use~\parencite{gonzales_amaya_delimiting_2018}.
This study used the ABEM~WalkTEM system with a $50\times50\,\text{m}$ transmitter loop with a current of $18\,\text{A}$ and
two receiver loops ($0.5\times0.5\,\text{m}$ and $10\times10\,\text{m}$) in a central configuration~\parencite{gonzales_amaya_delimiting_2018}.
This way a DOI of up to $200\,\text{m}$ was achieved in some regions of the alluvial fan, 
while in other regions the DOI was limited to $80\,\text{m}$~\parencite{gonzales_amaya_delimiting_2018}.
\textcite{gonzales_amaya_delimiting_2018} were able to detect zones with electrical resistivities of $0.1-1\,\Omega\text{m}$, 
which indicate high salinity (brine layer).
The results from the TEM soundings showed good agreement with borehole and electrical resistivity tomography (ERT) data 
and were even able to discern a potential fault in the bedrock not detected by the other methods~\parencite{gonzales_amaya_delimiting_2018}.

\textcite{gomez_alluvial_2019} conducted a similar study in the Challapampa area, Bolivia.
In this study the same WalkTEM system was used, yet the receiver loops were deployed in an offset configuration and
for the $0.5\times0.5\,\text{m}$ loop only a current of $2\,\text{A}$ was used, 
while the whole $18\,\text{A}$ were applied for the $10\times10\,\text{m}$ loop~\parencite{gomez_alluvial_2019}.
This setup achieved a depth of investigation (DOI) of up to $250\,\text{m}$
and identified the influence of a hot spring, which appeared as a low-resistivity zone ($5\,\Omega\text{m}$), 
indicating higher salinity~\parencite{gomez_alluvial_2019}.

Another application of TEM is the detection of karstic features, like caves, faults, and fracture zones~\parencite{zhou_multi-turn_2022, su_response_2024}.
Traditionally, a large transmitter loop size in the order of $100\,\text{m}$ side length was used to achieve a high investigation depth, 
which made the TEM method not suitable for the application in mountainous regions~\parencite{zhou_multi-turn_2022,su_response_2024}.
However, a similar investigation depth can be achieved by using a smaller transmitter loop size and more turns~\parencite{zhou_multi-turn_2022}.
The disadvantage beint that a multi-turn setup is affected by mutual inductance caused by the large number of turns, 
which can lead to underestimation of the resistivity of the subsurface~\parencite{zhou_multi-turn_2022}.

\textcite{zhou_multi-turn_2022} propose a coincident configuration with a $2\times2\,\text{m}$ loop size and 10 turns for the transmitter, 
and 20 turns for the receiver antenna using the TEM system ``YCS512'' by Fuzhou Huahong Intelligent Technology Co., Ltd., China.
The authors of this study did not disclose the acquisition time used for the soundings, 
but \textcite{chengshuai_research_2024} describe the YCS512 system to have a sampling time window of $9500\,\mu\text{s}$.
To mitigate the effects of the mutual inductance, borehole 
and ERT data were used to constrain the inversion~\parencite{zhou_multi-turn_2022}.
As boreholes and ERT measurements are expensive and time-consuming,
they were used at a single sounding location to quantify the shift in the resistivity values caused by 
mutual inductance, which was then used to correct the remaining TEM soundings~\parencite{zhou_multi-turn_2022}.
\textcite{zhou_multi-turn_2022} were able to discern an area with a electrical resistivity of $80\,\Omega\text{m}$ in the results
obtained through the TEM measurements, which indicate a potential underground karst channel.
The constrained inversion of the TEM data showed consistent agreement with real strata, 
but the abnornmal zones obtained through ERT and TEM showed minor discrepancies due to the differnent
underlying principles of the two methods~\parencite{zhou_multi-turn_2022}. 

A more in-depth study on how to deal with self and mutual inductance was conducted by \textcite{su_response_2024}.
Here a central loop configuration (receiver loop in the centre of the transmitter loop) with a $1.5\times1.5\,\text{m}$ transmitter loop and 6 turns 
was compared with a multi-turn small fixed-loop configuration
(using multiple different receiver positions, while keeping the transmitter loop fixed) with a $3\times3\,\text{m}$ transmitter loop 
and a single turn~\parencite{su_response_2024}.
The Pro-TEM47HP system was used for both configurations with a current of $1.5\,\text{A}$ 
and a turn-off time of $5-25\,\mu\text{s}$~\parencite{su_response_2024}.
For the fixed-loop set up a correction coefficient was introduced to account for off-centre receiver positions~\parencite{su_response_2024}.
\textcite{su_response_2024} conducted 5 model tests in a $5\times4\times3\,\text{m}$ sand-filled area
to compare the two configurations and the results showed that after correction the fixed-loop configuration
was more accurate in detecting the position of one or multiple anomalies.

\subsubsection{Common Electrical Resistivities of Subsurface Media}
\label{subsubsec:common-resistivity-values}

Electrical resisitvity (or conductivity), seismic velocities, and dielectric constand are the most 
relevant petrophysical properties used in geophysical investigations to identify the geological composition
of the subsurface~\parencite{kirsch_petrophysical_2006}.
The TEM method is mainly used to infer the electrical resistivity of the subsurface~\parencite{christiansen_transient_2006}, 
altough there are newer approaches to also characterise polarisation effects~\parencite{aigner_sensitivity_2024}.
By analysing the electrical resistivity values, it is possible to infer the composition of subsurface layers
and detect geological structures such as faults and fractures~\parencite{zhou_multi-turn_2022,su_response_2024}, 
and aquifers~\parencite{danielsen_application_2003, auken_ttem_2019, gonzales_amaya_delimiting_2018,gomez_alluvial_2019}.
\cref{tab:resistivity} presents commonly reported resistivity values for common geological media relevant to groundwater exploration.

\begin{table}[h]
    \centering
    \caption{Common Resistivity Values of Subsurface Materials}
    \resizebox{\columnwidth}{!}{
        \begin{tabular}{l c}
            \toprule
            \textbf{Material} & \textbf{Resistivity ($\Omega\text{m}$)} \\
            \midrule
            Saturated layers with high salinity & $0.1-5$ \\
            Saturated clays and silts & $5-15$ \\
            Saturated sediments & $10-20$ \\
            Unsaturated sediments & $50-80$ \\
            Saturated sand & $40-150$ \\
            Unsaturated sand & $400-1500$ \\
            \bottomrule
        \end{tabular}
    }
    \label{tab:resistivity}
\end{table}

These values have been compiled from multiple studies conducted in diverse geological settings and
using various geophysical methods~\parencite{gomez_alluvial_2019, galazoulas_large_2015, george_modelling_2022}.
In the context of~\cref{tab:resistivity}, the term ``sediments'' refers to unconsolidated deposits 
consisting of a mixture of sand, silt, and clay.
The precise resistivity values presented in~\cref{tab:resistivity} for sediments are not general global values but are
specific to the study region investigated by \textcite{gomez_alluvial_2019}.
These values reflect the characteristics of sediments found in the Challapampa aquifer in Bolivia,
where variations in grain size, moisture content, and mineral composition influence the resistivity measurements.

\subsubsection{Data Inversion}
\label{subsubsec:data-inversion}

\textcite{zhdanov_geophysical_2002} defines a typical geophysical problem can be defined as follows:

\begin{equation}
    \mathbf{d} = \mathcal{F}(\mathbf{m})
    \label{eq:geophysical_problem}
\end{equation}

where $\mathbf{d}$ is the observed data, $\mathcal{F}$ is the forward operator, and $\mathbf{m}$ is the model.
If the model is known, the forward operator can be used to calculate the expected data~\parencite{zhdanov_geophysical_2002}.
In geophysical investigations usually the model is unknown and the observed data is used to find the model, 
which is called an inverse problem~\parencite{xue_development_2020}.
In practise, the observed data is contaminated with noise, which introduces an error term into the equation~(\cref{eq:geophysical_problem})
and thus expands the number of possible models, which can explain the observed data 
and thus making the inverse problem in geophysics ill-posed and non-linear~\parencite{zhdanov_geophysical_2002}.
An ill-posed problem means that mathematically there is no or no unique solution~\parencite{zhdanov_geophysical_2002}.

Approaches to address this issue can be divided into deterministic and stochastic methods~\parencite{xue_development_2020}.
The deterministic approach tries to find a single solution, by iteratively updating the model parameters
to minimise the difference between observed and modeled data~\parencite{xue_development_2020}.
To prevent overfitting to noise, Tikhonov regularisation is used, 
which adds a penalty term to the least squares problem~\parencite{zhdanov_geophysical_2002}, such as:

\begin{equation}
    \| \mathbf{W}_d (\mathcal{F}(\mathbf{m}) - \mathbf{d}) \|_2^2 + \lambda \| \mathbf{W}_m (\mathbf{m} - \mathbf{m}_0) \|_2^2 \to \min
    \label{eq:tikhonov}
\end{equation}

$\mathbf{W}_d$ and $\mathbf{W}_m$ are weighting matrices, $\mathbf{m}_0$ is the initial model, 
and $\lambda$ is the regularisation parameter.
$\| \mathbf{W}_m (\mathbf{m} - \mathbf{m}_0) \|_2^2$ can be interpreted as the roughness of the model, 
which quantifies the complexity or variation of the model~\parencite{zhdanov_geophysical_2002}.
$\| \mathbf{W}_d (\mathcal{F}(\mathbf{m}) - \mathbf{d}) \|_2^2$ is the data misfit, 
which quantifies the difference between observed $\mathbf{d}$ and modeled $\mathcal{F}(\mathbf{m})$ data 
and can include an error term~\parencite{zhdanov_geophysical_2002}.
With the choice of $\lambda$, the trade-off between data misfit and model complexity can be controlled~\parencite{xue_development_2020}.
Building on this foundation, several deterministic methods have been developed, 
such as Gauss-Newton inversion and Conjugate Gradient~\parencite{xue_development_2020}.
A constraint inversion can be used to incorporate prior information about the model, 
which can be used to reduce the number of models able to explain the data,
and thus improve accuracy and reliability~\parencite{xue_development_2020}.

Stochastic methods, on the other hand, randomly search the solution space 
and provide a range of plausible models rather than a single deterministic solution~\parencite{aigner_stochastic_2025,xue_development_2020}.
This is significantly more computationally expensive, 
but can be more robust against noise and can provide uncertainty estimates~\parencite{aigner_stochastic_2025}.
Particle swarm optimization (PSO) and Bayesian inversion are examples of stochastic methods~\parencite{xue_development_2020}.

An implementation of the deterministic approach is the \texttt{PyGIMLi} library, introduced by \textcite{rucker_pygimli_2017},
which uses Gauss-Newton inversion to iteratively update the model parameters.
\texttt{PyGIMLi} is an open-source library written in Python and C++ and is designed for the inversion of geophysical data~\parencite{rucker_pygimli_2017}.
\texttt{PyGIMLi} allows the implementation of any given forward operator into the inversion algorithm,
which makes it a versatile tool for geophysical modeling and inversion~\parencite{rucker_pygimli_2017}.

\subsubsection{L-curve method}
\label{subsubsec:l-curve-method}

Solving inverse problems is a task not limited to geophysics, which makes inversion theory an important field in mathematics.
There are several methods to solve an inverse problem, but the most common approach is to use Tikhonov regularisation~\eqref{eq:tikhonov}.
But the choice of the regularisation parameter $\lambda$ is not trivial and can have a significant impact on the inversion result.

The L-curve method is widely used to determine the optimal $\lambda$ for the solution of an inverse problem~\parencite{hansen_analysis_1992}.
The L-curve is a graph of the residual norm against the solution norm.
With an increasing $\lambda$, the residual norm is expected to increase and the solution norm is expected to decrease.
This leads to a curve that resembles an L-shape.
An optimal $\lambda$ should minimise the residual norm while keeping the solution norm small.
This leads to the ``corner'' of the L, which is also the point with the highest curvature.

\textcite{lloyd_use_1997} implements a method to find the point of maximum curvature and through this the optimal $\lambda$ for the inversion of diffusion battery data.
The method computes the $\chi^2$, also called ``error weighted root-mean-square'', and the roughness of the model for different $\lambda$ values to obtain the L-curve.
A cubic spline function is fitted to the data points, to make it possible to calculate the curvature for each data point:

\begin{equation}
\begin{alignedat}{2}
    \mathbf{C}(\lambda_i) & = \frac{d^{2}s/dx^{2}}{(1 + (ds/dx)^2)^{3/2}} \\
    x & = \log_{10}{\chi^2(\lambda_i)}
\end{alignedat}
\label{eq:curvature}
\end{equation}

and find the point with the maximum curvature.
The corresponding $\lambda$ is then considered the optimal one.
This method was not developed for geophysical data, but a modell roughness and $\chi^2$ can be computed for the TEM data as well.

A similar approach was used by \textcite{farquharson_comparison_2004} to find the optimal $\lambda$.
In this method the value of the regularization parameter $\lambda$ is refined in each iteration of the data inversion.
The inversion is started with a large $\lambda$ and the L-curve is calculated.
Then the curvature for the chosen $\lambda$ is calculated through the formula:

\begin{equation}
\begin{alignedat}{2}
    \mathbf{C}(\lambda) & = \frac{\zeta'\eta''-\zeta''\eta'}{[(\zeta')^2 + (\eta')^2]^{3/2}} \\
    \zeta & = \log{\phi_d^{\text{lin}}} \\
    \eta & = \log{\phi_m}
\end{alignedat}
\label{eq:curvature_2}
\end{equation}

$\phi_d^{\text{lin}}$ is the data misfit and $\phi_m$ is the model roughness.
For the next iteration a new $\lambda$ is calculated based on:

\begin{equation}
    \lambda^{n} = \max(c\lambda^{n-1}, \lambda^{\max})
    \label{eq:lambda_new}
\end{equation}

where $\lambda^{n}$ is the new $\lambda$ value, $\lambda^{n-1}$ is the previous $\lambda$ value, $0.01 \leq c \leq 0.5$
and $\lambda^{\max}$ is the value for $\lambda$, which maximises the curvature.
This cooling-schedule-type behaviour is added to prevent the inversion to skip to low values of $\lambda$,
which is supposed to prevent artifacts created by overfitting the data.
This method was tested on synthetic frequency domain electromagnetic data and was able to achieve an appropriate fit of inverted to the observed data.

Another approach for finding the optimal $\lambda$ is the iterative golden section search as proposed by \textcite{cultrera_simple_2020}.
After providing an initial range for the optimal $\lambda$ $[\lambda_1,\lambda_4]$, two more $\lambda$ values are calculated using the formula:

\begin{equation}
\begin{alignedat}{2}
    \varphi & = \frac{1 + \sqrt{5}}{2} \\
    \lambda_2 & = 10^{\frac{\log_{10}{\lambda_4} + \varphi \cdot \log_{10}{\lambda_1}}{1 + \varphi}} \\
    \lambda_3 & = 10^{\log_{10}{\lambda_1} + (\log_{10}{\lambda_4} - \log_{10}{\lambda_2})}
\end{alignedat}
\label{eq:gls_search}
\end{equation}

For each $\lambda$ the corresponding point on the L-curve is found and two curvatures ($C_2$ and $C_3$) are computed relying on three points each.
$C_2$ is the curvature of the points $\lambda_1$, $\lambda_2$, and $\lambda_3$. $C_3$ is the curvature of the points $\lambda_2$, $\lambda_3$, and $\lambda_4$.
Then $\lambda_1$ or $\lambda_4$ is omitted depending on which curvature is larger and a fourth $\lambda$ is calculated based on the formula for $\lambda_2$~\eqref{eq:gls_search}.
This process is repeated until the difference between the lambda-values of the interval are below a certain threshold.
This method allows to find an optimal $\lambda$ while minimising the number of inversions necessary.
The search algorithm was tested with the ERT method on a conductive thin film with two non-conductive anomalies
and showed promising results in finding the ``corner'' of the L-curve.