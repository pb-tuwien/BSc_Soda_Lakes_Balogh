\section{State of the Art}

The transient electromagnetic (TEM) is a well established method for hydrogeophysical applications. For example, the exploration of groundwater aquifers or the detection of saltwater intrusion into the groundwater.
This methods works by injecting DC current into an transmitter loop and abruptly cutting the current flow. The resulting turn off ramp causes an electromotive force which causes currents in the ground (eddy currents). These currents in turn create a secondary magnetic field which induces a current in a receiver loop, which can be measured. This magnetic field will decay over time. As the decay is a function of time, this method is electromagnetic in the time domain.

At first, this decay is mainly influenced by the electromotive force of the transmitter coil, but in late times, the rate of this decay depends on the conductivity of the subsurface below the transmitter coil. As time continues, the electromagnetic waves travel deeper into the ground, which means that later times correspond to influences from further below.

\section{Materials and Methods}

\subsection{Test site}
The measurements for this thesis were gathered at Martenhofer Lakes located in the Nationalpark Seewinkel in Burgenland (\color{coordinates}). The Martenhofer Lakes consist of two lakes which belong to a group of soda lakes in this area which present a unique habitat for many organisms. This site was chosen because the smaller lake in the north east has already dried up, while the bigger lake still bears plenty of water. The measurements were done in late September right after an extreme precipation event in the east of Austria which resulted in many floods. Because of this the region beared more water than usuall at this time of the year. For this reason the smaller lake was partially filled, which is out of the ordinary.
Two campaigns were carried out with a separation of two weeks. First using ERT the lakes were mapped with multiple profiles. On the second campaign TEM was used to map the lakes with multiple soundings.

