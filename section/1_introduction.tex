As environmental applications of subsurface exploration gain importance,
geophysical methods, which allow for a non-invasive shallow investigations, 
become more relevant~\parencite{barsukovChapter3Shallow2006}.
The transient electromagnetic (TEM) method is a time-domain electromagnetic technique
widely used for exploring the subsurface, providing valuable insights for groundwater detection~\parencite{danielsenApplicationTransientElectromagnetic2003,aukenMutuallyConstrainedInversion2005},
monitoring of saline water intrusions~\parencite{gomezAlluvialAquiferThickness2019,gonzalesamayaDelimitingSalineWater2018}, 
and detection of karstic features like caves~\parencite{zhouMultiturnSmallloopTransient2022,suResponseAnalysisApplication2024}.

Interpreting TEM data effectively requires the solution of an ill-posed inverse problem,
which means to calculate a model of the subsurface resistivities from the observed data 
with the limitation that various models exist, which can explain the measured data~\parencite{zhdanovGeophysicalInverseTheory2002}.
Inversion techniques have been developed, which employ a smoothness-constraint approach, 
where the data-model misfit is balanced by a smoothing operator (weighted with a regularisation parameter) that helps to obtain a smooth solution, 
which has a closer correlation with expected (smooth) geological changes~\parencite{ruckerPyGIMLiOpensourceLibrary2017}.
This requires choosing a value for the regularisation parameter ($\lambda$), 
which balances the fit of the data with the complexity of the model~\parencite{zhdanovGeophysicalInverseTheory2002}.
Ill-posed inverse problems are not unique to interpretation of TEM data and 
thus the L-Curve method was developed in other fields to find an optimal value for the 
$\lambda$~\parencite{hansenAnalysisDiscreteIllPosed1992,cultreraSimpleAlgorithmFind2020,lloydUseLcurveMethod1997}.

This thesis aims to adapt the L-Curve method for the interpretation of TEM data, 
to improve the inversion results through the selection of an adequate regularization parameter.
The hypothesis is, that the L-curve method can be used to identify the optimal $\lambda$ (regularisation parameter) 
for the inversion of TEM data and thus improve the obtained results.
To investigate this, data was collected at the Martenhofer Lacke
as part of the effort of the TU Wien Department of Geodesy and Geoinformation,
Research Unit Geophysics, to understand the hydrogeological processes
of the soda lakes in the Nationalpark Neusiedlersee - Seewinkel.
The soda lakes pose a unique habitat for various species, which is threatened by climate change, 
making it important to grasp the processes governing the ecosystem~\parencite{borosHowCanUnique2025}.

In paticular the objective of this work is to adapt the inversion routine proposed by \textcite{aignerFlexibleSingleLoop2021},
to compute the L-Curve, which can be visually analysed to find an optimal $\lambda$, 
as well as three automatic search algorithms based on the works 
of \textcite{cultreraSimpleAlgorithmFind2020,lloydUseLcurveMethod1997}.
This work investigates, what conditions are necessary for the automatic determination of an optimal $\lambda$ value, 
as well as which automated search algorithms can be implemented to find an optimal $\lambda$.
This is achieved by comparing the $\lambda$ values returned by the search algorithms with values obtained 
through a visual investigation of the computed L-Curve plot.

For this thesis TEM data was collected in two different survey at the Martenhofer Lacke, which were four month apart, 
with two different loop sizes and two different currents used for the measurements.
As this data set allows for a comparison between the measuring configurations,
this work also investigates, what TEM configuration is better suited to investigate 
the shallowest three layers underlying the Martenhofer Lacke.
This is achieved by analysing the inversion results and comparing them with previous studies
in the area~\parencite{aignerSensitivityAnalysisInverted2024,aignerStochasticInversionTransient2025}.
