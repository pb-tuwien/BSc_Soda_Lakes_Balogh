near surface
environmental applications
geophysical methods
non-invasive exploration

As environmental applications of subsurface exploration gain importance,
geophysical methods, which allow for a non-invasive shallow investigations, 
become more relevant~\parencite{barsukovChapter3Shallow2006}.
The transient electromagnetic (TEM) method is a time-domain electromagnetic technique
widely used for exploring the subsurface, providing valuable insights for groundwater detection~\parencite{danielsenApplicationTransientElectromagnetic2003,aukenMutuallyConstrainedInversion2005},
monitoring of saline water intrusions~\parencite{gomezAlluvialAquiferThickness2019,gonzalesamayaDelimitingSalineWater2018}, 
and detection of karstic features like caves~\parencite{zhouMultiturnSmallloopTransient2022,suResponseAnalysisApplication2024}.

Interpreting TEM data effectively requires the solution of an ill-posed inverse problem,
which means to calculate a model of the subsurface resistivities from the observed data 
with the limitation that various models exist, which can explain the measured data~\parencite{zhdanovGeophysicalInverseTheory2002}.
Inversion techniques have been developed, which employ a smoothness-constraint approach, 
where data-model misfit is balanced by a smoothing operator (weighted with a regularisation parameter) that helps to obtain a smooth solution, 
which has a closer correlation with expected (smooth) geological changes~\parencite{ruckerPyGIMLiOpensourceLibrary2017}.
This requires choosing a value for the regularisation parameter ($\lambda$), 
which balances the fit of the data with the complexity of the model~\parencite{zhdanovGeophysicalInverseTheory2002}.
Ill-posed inverse problems are not unique to interpretation of TEM data and 
thus the L-Curve method was developed in other fields to find an optimal value for the 
$\lambda$~\parencite{hansenAnalysisDiscreteIllPosed1992,cultreraSimpleAlgorithmFind2020,lloydUseLcurveMethod1997}.

This thesis aims to adapt the L-Curve method for the interpretation of TEM data, 
to improve the inversion results through the selection of an adequate regularization parameter.
For this purpose data was collected at Martenhofer Lacke
as part of the effort of the TU Wien Department of Geodesy and Geoinformation,
Research Unit Geophysics, to understand the hydrogeological processes
of the soda lakes in the Nationalpark Neusiedlersee - Seewinkel.
The soda lakes pose a unique habitat for various species, which is threatened by climate change, 
making it important to grasp the processes governing the ecosystem~\parencite{borosHowCanUnique2025}.
In paticular the objective of this work is to adapt the inversion routine proposed by \textcite{aignerFlexibleSingleLoop2021},
to compute the L-Curve, which can be visually analysed to find an optimal $\lambda$, 
as well as three automatic search algorithms based on the works 
of \textcite{cultreraSimpleAlgorithmFind2020,lloydUseLcurveMethod1997}.

\colorbox{yellow}{Finished until here}

The hypothesis is, that the L-curve method can be used to identify the optimal $\lambda$ (regularisation parameter) 
in the inversion TEM data and thus improve the obtained results.

Research questions:
\begin{itemize}
    \item Which conditions are necessary for the determination of an optimal $\lambda$ value?
    \item Which automated search algorithms can be implemented to find an optimal $\lambda$?
    \item Which configuration of the TEM method is suitable for the investigation of the Martenhofer Lacke?
\end{itemize}
