Introduction including:

% Objective: Create a model of the subsurface resistivity at Martenhofer Lacke using the transient electromagnetic method.

% Hypothesis: An optimal lambda can be found for the TEM data gathered at Martenhofer Lacke by using the L-Curve method.

Research questions:
\begin{itemize}
    \item Which conditions are necessary for an automatic determination of the lambda value?
    \item Which automated search algorithms can be implemented to find an optimal lambda?
    \item Which configuration of the TEM method is suitable for the investigation of the Martenhofer Lacke?
\end{itemize}

Hypothesis: The L-curve is a method that can be used to identify the best lambda (smoothing parameter) 
in the inversion and improve the results of the TEM inversion.

Objective: Improve the inversion of TEM data through the selection of an adequate regularization parameter.

\colorbox{yellow}{Adrians start:}
In this thesis I present the inversion of transient em data collected at the Martenhofer lacke. 
The movitvation is to better delineate the geometry of the materials controlling the surface-groundwater 
interactions of the lacke. 
In particular, I aim at investigating the geometry of the three 
hydrogeological layer defining the lacke.

The inversion of TEM data is affected by ill-posedness, 
which means different models can explain the data, 
and small uncertainty in the data results large uncertainties in the inverted model. 
To overcome this, inversion of geophysical data commonly uses a smoothness-constraint approach, 
where data-model misfit is balanced by a smoothing operator that helps to obtain a smooth solution, 
which has a closer correlation with expected (smooth) geological changes.

\colorbox{orange}{Possible start:}
The transient electromagnetic (TEM) method is a time-domain electromagnetic technique
widely used for exploring the subsurface, providing valuable insights into environmental
applications like groundwater detection~\parencite{aigner_sensitivity_2024}. % TODO: check if good source
Interpreting TEM data effectively requires accurate inversion techniques of the data,
and the L-curve method stands out for helping balance the fit of the data with the complexity of the model.
