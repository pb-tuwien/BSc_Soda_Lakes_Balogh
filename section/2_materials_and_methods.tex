\label{sec:materials_and_methods}
% \clearpage
\subsection{State of the Art}\label{subsec:state-of-the-art}
\subfile{2_1_state_of_the_art}

\subsection{Experimental Set up}\label{subsec:set-up}
The L-Curve method was implemented for the inversion of TEM data and tested in a study site.

\subsubsection{Measuring Device}\label{subsubsec:device}
The TEM measurements for the field survey were done with the TEM-Fast 48 system by Applied Electromagnetic Research (AEMR). It is compact device allowing the use of a single-loop configuration. By connecting an external $12~/~24\,V$ battery a current either $1~/~4\,A$ can be put through the connected transmitter loop.
It records up to 48 logarithmically spaced time gates, which results it a time range between $4 - 16000\,\mu s$. The specific number of time gates can be chosen through a time-key. \cref{tab:time-key} shows which time-key leads to which recording time range.
The system allows for the stacking of pulses. The number of stacks are given by the formula $P_{tot} = 13 \times n_s \times n_{as}$~\parencite{aigner2021flexible}, where $n_s$ ($1-20$) is the chosen stacking-key and $n_{as}$ is the number of analogue stacks depending on the chosen time-key and can be found in \cref{tab:time-key}.


\begin{table}[h]
    \centering
    \caption{Parameters relating to the time-key of the TEM-FAST 48 system (Excerpt from the manual)}
    \resizebox{\columnwidth}{!}{
    \begin{tabular}{cccc}
        \hline
        \textbf{Key} & \textbf{Max Time ($\mu s$)} & \textbf{Active Time Gates} & \textbf{Analog Stacks} \\
        \hline
        1 & 64 & 16 & 1024\\
        2 & 128 & 20 & 512\\
        3 & 256 & 24 & 256\\
        4 & 512 & 28 & 128\\
        5 & 1024 & 32 & 64\\
        6 & 2048 & 36 & 32\\
        7 & 4096 & 40 & 16\\
        8 & 8192 & 44 & 8\\
        9 & 16384 & 48 & 4\\
        \hline
    \end{tabular}
    }
    \label{tab:time-key}
\end{table}

\subsubsection{Field Survey}\label{subsubsec:field-survey}
The field measurements were carried out at the Martenhofer Lacke in the \textit{Nationalpark Neusiedlersee - Seewinkel} (16° 51' 23.058'' N, 47° 45' 8.4348'' E).
This site is located on the east side of the Neusiedler See, Austria.
The Martenhofer Lacke belongs to the soda lakes, which are characterised by a high salt content in the water.
Because this site is part of the national park, it can be assumed that it is a low noise environment.
For this work, two separate surveys were conducted: One on the 22nd May 2024 with 45 soundings and a second one on 8th October 2024 with 66 soundings.
Both surveys were done with the TEM-FAST 48 system manufactured by Applied Electromagnetic Research (AEMR) with a single-loop configuration.
In the survey in May, a $12.5\times12.5\,\text{m}$ loop was combined with a current of $1.0\,\text{A}$ for all soundings except for the first two (M001 and M002), where $4.1\,\text{A}$ were used.
For the first 14 soundings (M001-M014), 28 time windows were observed, resulting in a time range of $4-480\,\mu\text{s}$, and 4992 stacks.
For all other soundings (M015-M045), 24 time windows were observed, resulting in a time range of $4-240\,\mu\text{s}$, and 9984 stacks.
In the survey in October, a $6.25\times6.25\,\text{m}$ loop was used with a current of $4.1\,\text{A}$.
For all soundings, a time range of $4-240\,\mu\text{s}$(24\,\text{time windows}) were combined with 16640 stacks.


\subsubsection{Inversion Algorithm}\label{subsubsec:inversion}
hi