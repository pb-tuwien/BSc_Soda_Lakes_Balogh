\label{sec:materials_and_methods}

\subsection{State of the Art}\label{subsec:state-of-the-art}


\subsubsection{Common resistivity values}\label{subsubsec:common-resistivity-values}

\subsubsection{TEM method}\label{subsubsec:tem-method}

\subsubsection{Application of TEM}\label{subsubsec:application-of-tem}
Even though the TEM method originates from the mining industry, it can be applied to various environmental applications.
The exploration and characterisation of groundwater resources is one of the most common applications of the TEM method.

For this application \textcite{danielsen2003application} used the TEM method for the exploration of buried valleys as potential aquifers in Denmark.
The study introduced two new TEM systems based on the Geonics PROTEM, the High moment Transient Electromagnetic (HiTEM) system for a deeper investigation depth
and the Pulled Array Transient Electromagnetic (PATEM) system for a higher lateral resolution.
The HiTEM system uses a combination of a $30\times30 m$ transmitter loop and a current of 75 A to achieve a depth of investigation (DOI) of up to 300 m.
To mitigate distortion effects an offset receiver loop is used for the later times and a central receiver loop with a reduced current of 2.5 A for the early times.
A Mutually Constrained Inversion (MCI) was used to invert the combination of the data gathered with the two different configurations.
The PATEM system uses a $3\times5 m$ transmitter loop on a wheeled frame with an offset receiver loop.
This allows for continuous measurements along a profile and a DOI of 100 to 150 m.
To achieve a high DOI and useful near surface information, the PATEM system allows for a transmitter configuration with either 2 turns with 16 A or 8 turns with 40 to 50 A\@.
The study also touches on the problems resulting from coupling with man-made structures.
The coupling effects can be divided into galvanic and capacitive.
Galvanic coupling is caused by grounded conductor like a power line and causes the underestimation of the resistivity.
Capacitive coupling is caused current being generated in a conductor and leaking into the ground through an insulation, which leads to an oscillating signal.
For this reason the study recommends to keep a distance of about 150 m from underground cables or pipes when the earth has a resistivity of 40 to 80 $\Omega m$.
The study finds that a 1D inversion approach is sufficient to detect the slope of a 2D buried valley when in a layer with a low resistivity.
In a layer with a high resistivity only the overall structure of the valley can be derived.

Similar to the PATEM system \textcite{auken2019ttem} proposes a towed Transient Electromagnetic (tTEM) system for an efficient 3D mapping of the subsurface.
The tTEM system utilizes a $2\times4 m$ transmitter loop mounted on a non-conductive sled towed by an ATV, which enables a production rate of about $1 km^2$.
The receiver loop is towed at a 9 m offset to the transmitter.
Like the PATEM system the tTEM system allows two different currents (2.8/30A) in order to gain a relatively high DOI of up to 70 m,
while still allowing the investigation of shallow depths.
This study also highlights the importance of the coupling effects with conductive objects
and tested the system in an environment with a high resistivity of the subsurface (>600 $\Omega m$).
Here the measured signal can be assumed as a coupling effect with the conductive objects on the ATV and a minimal distance of 3 m was determined to be necessary.
As it is important o calibrate TEM systems, the tTEM system was validated at the Danish National TEM test site.
Furthermore, the tTEM system was also validated against borehole data and the results showed a good agreement.

As other water sources become less reliable \textcite{gonzales2018delimiting}
showed the potential of the TEM method for the exploration of groundwater in the Punata alluvial fan, Bolivia.
This aquifer poses an important water source, but it has zones with high salinity.
This study used the ABEM WalkTEM system with a $50\times50 m$ transmitter loop with a current of 18 A and
two receiver loops ($0.5\times0.5 m$ and $10\times10 m$) in a central configuration.
This way a DOI of up to 200 m was achieved in some regions of the alluvial fan, while in other regions the DOI was limited to 80 m.
The study was able to detect zones with high salinity and the results were validated with borehole data.

A similar study was conducted by \textcite{gomez2019alluvial} in the Challapampa area, Bolivia.
In this study the same WalkTEM system and loop sizes were used as in the study by Gonzales et al.\cite{gonzales2018delimiting}
But the receiver loops were deployed in an offset configuration and
for the $0.5\times0.5 m$ loop only a current of 2 A was used, while the whole 18 A were used for the $10\times10 m$ loop.
With this configuration a DOI of up to 250 m was achieved and the influence of a hot spring was detected as a low resistivity zone (meaning higher salinity).

Another application of TEM is the detection of karstic features, like caves, faults, and fracture zones\parencite{zhou2022multi, su2024response}.
Traditionally a large transmitter loop size in the order of 100 m side length was used to achieve a high investigation depth.
For this reason the TEM method was not suitable for the application in mountainous regions.
However, a similar investigation depth can be achieved by using a smaller transmitter loop size and more turns.
A multi-turn setup has the disadvantage of mutual inductance caused by the large number of turns, which can lead to underestimation of the resistivity of the subsurface.

\textcite{zhou2022multi} proposed a coincident configuration with a 2 m side length and 10 turns for the transmitter loop and 20 turns for the receiver loop.
In order to mitigate the effects of the mutual inductance, borehole and electrical resistivity tomography (ERT) data were used to constrain the inversion.
Because boreholes and ERT measurements are expensive and time-consuming, they are only used to correct to TEM soundings in order to account for shifted resistivity values.
This method was used to detect a karst channel in Zhijin, China.

A more in depth study on how to deal with self and mutual inductance was conducted by \textcite{su2024response}.
Here a central loop configuration (receiver loop in the center of the transmitter loop) was compared with a multi-turn small fixed-loop configuration,
where multiple different receiver positions are used, while keeping the transmitter loop fixed.
For the fixed-loop set up a correction coefficient was introduced to account for off centre receiver positions.
5 model test were conducted to compare the two configurations and the results showed that after correction the fixed-loop configuration
was more accurate in detecting the position of one or multiple anomalies.







\subsubsection{Data Inversion}\label{subsubsec:data-inversion}


\subsubsection{L-curve method}\label{subsubsec:l-curve-method}

A common way to solve an ill-posed problem is to use Tikhonov regularization, which is a method that adds a penalty term to the least squares problem.
The penalty term is a function of the model parameters and a regularization parameter lambda.
...

One way to determine an optimal values for the regularization parameter $\lambda$ is the L-curve method as introduced by Hansen\cite{hansen1999curve}.
The L-curve is a graph of the residual norm against the solution norm.
With an increasing lambda, the residual norm is expected to increase and the solution norm is expected to decrease.
This leads to a curve that resembles an L-shape.
The optimal $\lambda$ is the point on the curve where the curvature is the highest.
The L-curve method is widely used in geophysics for the determination of the optimal $\lambda$ for the inversion of geophysical data.

There are a several methods to automatically determine the optimal lambda as described in\cite{cultrera2020simple, farquharson2004comparison, lloyd1997use}.
LLoyd et al. \cite{lloyd1997use} proposed a method that computes the $\chi^2$, also called error weighted root-mean-square, and the roughness of the model for different $\lambda$a values to obtain the L-curve.
In order to find the optimal $\lambda$ a cubic spline function is fitted to the data points and used to find the maximum curvature of the L-curve.

Another approach is the iterative golden section search as proposed by Cultrera\cite{cultrera2020simple}.
After providing an initial range for the optimal lambda $[\lambda_1,\lambda_4]$, two more lambda values are calculated using the formula:
\begin{equation}
    \varphi = \frac{1 + \sqrt{5}}{2}
\end{equation}
\begin{equation}
    \lambda_2 = 10^{\frac{\log_{10}{\lambda_4} + \varphi \cdot \log_{10}{\lambda_1}}{1 + \varphi}}
\end{equation}
\begin{equation}
    \lambda_3 = 10^{\log_{10}{\lambda_1} + (\log_{10}{\lambda_4} - \log_{10}{\lambda_2})}
\end{equation}
For each $\lambda$ the corresponding point on the L-curve is found and two curvatures are computed relying on three points each.
The borders of the search interval are shifted towards the higher curvature and this process is repeated until the difference between the lambda-values of the interval are below a certain threshold.
This method allows can find an optimal $\lambda$ while minimizing the number of inversions necessary.
